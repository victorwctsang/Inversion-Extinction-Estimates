\section{Bias of Maximum Likelihood Estimator}

\section{Strauss Estimator Proof}

\section{Stochastically Increasing Property of MINMI}
\clearpage

\section{Asymptotic Variance of MINMI}\label{apx: minmi-asymptotics}

We want to find the asymptotic variance for our estimator $\hat\theta_q$, which is the quantile estimate of the sample quantile $\theta_q$. Recall \autoref{eqn: minmi-ee}: \[ q^{1/n} = 1 - \frac{F(m-\hat\theta_q)}{F(K-\hat\theta_q)} \int^{m-\hat\theta_q}_{-\infty} \frac{m-e-\hat\theta_q}{K-e-\hat\theta_q} \frac{f(e)}{F(m-\hat\theta_q)} de \]

Let $\psi(\theta) = \int^{m-\theta}_{-\infty} \frac{m-e-\theta}{K-e-\theta} \frac{f(e)}{F(m-\theta)} de$ and let $\hat\psi(\theta) = \frac{1}{B} \sum_{b=1}^B \frac{m-e_b-\theta}{K-e_b-\theta}$ denote the Monte Carlo approximation of $\psi(\theta)$, where $e_b$ are sampled from density function $f$ truncated at $m-\theta$.

Now, rearranging \autoref{eqn: minmi-ee}, we can obtain the estimating equation $u(\theta)$ \begin{equation}
    u(\theta) = \frac{F(m-\theta)}{F(K-\theta)} \psi(\theta) - 1 + q^{1/n}
\end{equation} such that $\theta_q$ satisfies $u(\theta_q) = 0$. By substituting $\psi$ with its Monte Carlo approximation $\hat\psi$, we can obtain $\hat{u}(\theta)$ \begin{equation}
    \hat{u}(\theta) = \frac{F(m-\theta)}{F(K-\theta)} \hat{\psi}(\theta) - 1 + q^{1/n}
\end{equation} such that $\hat\theta_q$ satisfies $\hat{u}(\hat\theta_q) = 0$.

Next, we consider the asymptotic distribution of $\hat\psi(\theta)$. By the Central Limit Theorem, we have \begin{equation} \label{eqn: psi_hat_CLT}
    \sqrt{B} (\hat\psi(\theta) - \psi(\theta)) \Dconverge \cN\left(0, \sigma^2_{\psi(\theta)} \right)
\end{equation} and by extension we have \begin{equation}
    \sqrt{B} (\hat{u}(\theta) - u(\theta)) \Dconverge \cN\left(0, \sigma^2_{\psi(\theta)} \left[ \frac{F(m-\theta)}{F(K-\theta)} \right]^2 \right)
\end{equation}

Since $u(\theta_q) = 0$, substituting $\theta = \theta_q$ we obtain \begin{equation}
    \sqrt{B} \hat{u}(\theta_q) \Dconverge \cN \left(0, \sigma^2_{\psi(\theta_q)} \left[ \frac{F(m-\theta_q)}{F(K-\theta_q)} \right]^2 \right)
\end{equation}

Next, we apply a Taylor expansion to $\hat{u}(\hat\theta_q)$ about $\hat{u}(\theta_q)$ \begin{align*}
    \hat{u}(\hat\theta_q) &\approx \hat{u}(\theta_q) + (\hat\theta_q - \theta_q) \hat{u}^\prime(\theta_q) \\
    \implies \hat\theta_q - \theta_q &\approx -\frac{\hat{u}(\theta_q)}{\hat{u}^\prime(\theta_q)} \quad \text{since $\hat{u}(\hat\theta_q) = 0$} \numberthis \label{eqn: taylor-expansion}
\end{align*}

$\hat{u}^\prime(\theta)$ is the Monte Carlo approximation of $u^\prime(\theta)$:\begin{align*}
    u(\theta) &= \frac{1}{F(K-\theta)} \int^{m-\theta}_{-\infty} \frac{m-e-\theta}{K-e-\theta} f(e) de - 1 + q^{1/n} \\
    u^\prime(\theta)
        &= \frac{f(K-\theta)}{\left[F(K-\theta)\right]^2} \int^{m-\theta}_{-\infty} \frac{m-e-\theta}{K-e-\theta} f(e) de + \frac{1}{F(K-\theta)} \int^{m-\theta}_{-\infty} \frac{\del}{\del \theta} \frac{m-e-\theta}{K-e-\theta} f(e) de \\
        &= \frac{f(K-\theta)}{\left[F(K-\theta)\right]^2} \int^{m-\theta}_{-\infty} \frac{m-e-\theta}{K-e-\theta} f(e) de + \frac{1}{F(K-\theta)} \int^{m-\theta}_{-\infty} \frac{m-K}{(K-e-\theta)^2} f(e) de \\
        &= \frac{F(m-\theta)}{F(K-\theta)} \left[ \frac{f(K-\theta)}{F(K-\theta)} \psi(\theta) +\psi^\prime(\theta) \right] \\
    \therefore \hat{u}^\prime(\theta) &= \frac{F(m-\theta)}{F(K-\theta)} \left[ \frac{f(K-\theta)}{F(K-\theta)} \hat\psi(\theta) +\hat\psi^\prime(\theta) \right]
\end{align*}

Since $\hat\psi(\theta)$ and $\hat\psi^\prime(\theta)$ are Monte Carlo integrals, we have $\hat\psi(\theta) \Pconverge \psi(\theta)$ and $\hat\psi^\prime(\theta) \Pconverge \psi^\prime(\theta)$. Since $\hat{u}^\prime$ is a linear function of $\hat\psi$ and $\hat\psi^\prime$, the continuous mapping theorem implies that $\hat{u}^\prime(\theta) \Pconverge u^\prime(\theta)$.

Now, applying Slutsky's Theorem to $\hat{u}(\theta) \Dconverge u(\theta)$ and $\hat{u}^\prime(\theta) \Pconverge u^\prime(\theta)$, we have $-\frac{\hat{u}(\theta_q)}{\hat{u}^\prime(\theta_q)} \Dconverge -\frac{{u}(\theta_q)}{{u}^\prime(\theta_q)}$ and therefore \begin{equation}
    \sqrt{B}(\hat\theta_q - \theta_q) \sim \cN \left(0, \sigma^2_{\psi(\theta_q)} \left[ \frac{F(m-\theta_q)}{F(K-\theta_q)\hat{u}^\prime(\theta_q)} \right]^2 \right)
\end{equation}
