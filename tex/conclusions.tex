
In this thesis, we proposed two novel applications of test-statistic inversion to estimate extinction times in the fossil record: the Minimum Statistic Inversion (MINMI) estimator and the Simulated Inversion - Robbins Monro Process (SI-RM). We explored these methods under assumptions of uniformly distributed fossils with both negligible measurement error and substantial measurement error, performing simulation studies to explore the performance of the obtained point estimates and confidence intervals in relation to existing methods.

\section{Discussion}

Most previously developed methods assumed uniformity and negligible measurement error, and our methods are a significant advance on this. We further expect that we could extend our methods to arbitrarily distributed fossils and measurement errors, using inversion to make exact inference (under monotonicity conditions).

In our simulation studies, we showed that the proposed methods produce point estimates that are comparable to existing methods, and can construct confidence intervals with better coverage, more appropriate widths, and faster computation times compared to existing methods such as GRIWM and SI-UGM. We then applied the MINMI and SI-RM estimators to datasets of megafauna fossils (the steppe bison, cave bear, cave hyena, and Eurasian woolly mammoth) and observed similar results to the simulation studies. We found that the proposed methods generated comparatively wider confidence intervals, which reflect the uncertainty around the true extinction time.

We found that a significant advantage of the MINMI estimator is its minimal tuning. Even as measurement error changed, it produced estimates that were comparable to SI-RM and SI-UGM without requiring significant adjustments to the approach. On the other hand, although SI-RM and SI-UGM are more flexible due to their use of a simulation model, they require decision-making that varies from application to application. As such, they are likely to be less accessible and more difficult to use than MINMI. There remains work to be done in developing software packages for palaeontologists to use MINMI and SI-RM; although some \texttt{R} code implementing the MINMI and SI-RM estimators has been provided in \autoref{apx:code-minmi} and \autoref{apx:code-si-rm}.

A key assumption made throughout this thesis is that fossils are uniformly distributed, and an opportunity going forward is to relax this assumption. Although it is common practice to assume uniformity, there is substantial evidence to suggest this is not valid as we can expect a species to be less abundant closer to speciation and extinction \cite{Lee2010, WangMarshall2016}. As a result, it would be worthwhile to explore the robustness of the proposed methods to alternative models, or adjust their design accordingly. For instance, one approach may be to use recovery rates from other fossil datasets' to estimate the distribution via kernel density estimation. Other suggestions include modelling fossil recovery as sigmoidal or exponentially distributed as per \citet{Bradshaw2012}, or using a ``reflected beta distribution" as per \citet{Wang2016}.

Unlike many existing methods, the proposed SI-RM estimator is more flexible and can be adapted to alternative models, as the simulation model simply needs to be adjusted accordingly. On the other hand, although the MINMI estimator can be adapted to a non-uniform model, the quantile function will need to be re-derived as currently this method's advantages are related to the exploitation of the conditionally uniform distribution of fossils.


\subsubsection{Advantages}

\begin{itemize}
    \item Exact under $\varepsilon$-model and stochastically increasing assumption.
    \item Very fast
\end{itemize}

\subsubsection{Limitations}

\begin{itemize}
    \item Choice of $B$ --- \autoref{theorem: delta-method-variance} is only true for large $B$, so \autoref{eqn:minmi-optimal-b} may sometimes give small values of $B$. We suggest being conservative; take the larger of \autoref{eqn:minmi-optimal-b} and $100$.
    \item Relies on uniformity assumption, results may not hold up under a different fossil distribution.
    \item Assumes measurement errors are IID; would be good to have a multivariate $f(e)$.
\end{itemize}

\subsubsection{Advantages}

\begin{itemize}
    \item Asymptotically exact under $\varepsilon$-model and stochastically increasing assumption.
    \item Flexible, since simulation model/step-sizes/starting-values/stopping criteria can be tuned.
    \item Faster than normal simulated inversion since it uses stochastic approximation to strategically select values of $\theta$ to sample at.
\end{itemize}

\subsubsection{Limitations}

\begin{itemize}
    \item Assumes that a good point estimate is available ($\hat\theta(\bm{x})$ is an ``attractor").
    \item Requires quite a lot of tuning
        \begin{itemize}
            \item Simulation model needs to be chosen
            \item Optimal step size needs to be estimated (dependent on the gradient of the quantile function). Can be tuned.
            \item Starting values significantly affects convergence.
            \item No general approach stopping criteria available; heuristics/inspection/cheating required.
        \end{itemize}
\end{itemize}

\section{Future Work}

A common assumption in the literature that has been carried over into this thesis is the independence of measurement error. Since radiocarbon ages are often calibrated using the same calibration curves, it is possible for the measurement errors to be correlated. This is known as a ``curve offset" \cite{Ramsey2010}. In his master's thesis, \citet{King2020} proposed an application of simulated inversion to account for calibration errors. We expect these to work similarly for SI-RM by adapting the simulation model accordingly. On the other hand, more work needs to be explore how the MINMI estimator can account for correlated calibration error.

In addition, it remains to be seen in general that the quantile function under the \hyperref[model: measurement-error]{$\varepsilon$-model} is stochastically increasing in $\theta$, an important condition for the application of test-statistic inversion. Although we were able to numerically verify this condition as well as show it is satisfied under some additional constraints (see \autoref{apx:minmi-stoch-incr-proof}), this remains an area of investigation for the future that may benefit the development of estimation methods based on test-statistic inversion.

Although the performance of our proposed methods is fairly good, a potential area of improvement is the choice of the test-statistic. The MINMI estimator uses the minimum statistic for inversion in the measurement error scenario, which was chosen because it is the MLE under the no-measurement error scenario. A better statistic could potentially be found, which may further improve estimates although the degree of improvement is uncertain. This is also true for the SI-RM estimator, which uses the test-statistic to select adaptive step lengths.

The SI-RM estimator may be improved across several aspects, such as the choice of step-length constant and stopping criteria. The SI-RM estimator uses the recommendations made by \citet{Garthwaite1992}, who proposed a heuristic that gives asymptotically exact results where the distribution of the point estimate is similar to a normal distribution, which could be further refined: for instance, \citet{LlyodBotev2015} suggested an alternative step-length constant that could be used to refine the step length constant following a burn-in period. Furthermore, additional work needs to be done to identify appropriate stopping criteria for the RM process, as stopping too early results in poor coverage and incorrect estimates.

In summary, the MINMI and SI-RM estimators proposed in this thesis are competitive alternatives to what currently exists for estimating extinction times. These methods produce comparable point estimates and significantly better confidence intervals compared to existing methods, and can be readily adapted to less strict assumptions which other methods are unable to do. These methods can also be further extended and improved across various areas to estimate extinction times under less convenient assumptions.

Additional `R` scripts and notebooks for this thesis are available online on GitHub: \url{https://github.com/victorwctsang/Inversion-Extinction-Estimates}