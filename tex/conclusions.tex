
In this thesis, we proposed two novel applications of test-statistic inversion to estimate extinction times in the fossil record: the Minimum Statistic Inversion (MINMI) estimator and the Simulated Inversion - Robbins Monro Process (SI-RM). We explored these methods under assumptions of uniformly distributed fossils with both negligible measurement error and substantial measurement error, performing simulation studies to explore the performance of the obtained point estimates and confidence intervals in relation to existing methods.

In our simulation studies, we found that both of the proposed methods produced estimates that were fairly comparable to existing methods. Although MINMI produced estimates that were somewhat comparable to existing methods, point estimates found using SI-RM (which was designed with confidence interval construction in mind) are identical to the point estimator, which we set to the MLE. These results were consistent across varying degrees of measurement error variation.

We also found that the proposed MINMI and SI-RM methods produced confidence intervals with coverage probabilities that were very close to nominal and with widths that better represent the uncertainty associated with extinction time estimates. These methods outperformed GRIWM, a commonly-used method proposed by \citet{Bradshaw2012} and the only other method that accounts for measurement error. MINMI and SI-RM estimates were also fairly similar to those found using the simulated inversion approach proposed by \citet{Huang2019} while improving the computational time significantly.

Most previously developed methods assumed uniformity with either negligible measurement error (the \hyperref[model: no-measurement-error]{$\delta$-model}) or substantial measurement error (\hyperref[model: measurement-error]{$\varepsilon$-model}), and our methods are a significant advance under these two models. We further expect that we could extend our methods to arbitrarily distributed fossils and measurement errors, using inversion to make exact inference under monotonicity conditions.

\section{Discussion}

The proposed MINMI estimator's most prominent advantage is its exactness under both the \hyperref[model: no-measurement-error]{$\delta$-model} and the \hyperref[model: measurement-error]{$\varepsilon$-model}. MINMI produces estimates that are exact under some monotonicity conditions and when using sufficiently many Monte Carlo samples.

A second advantage of the MINMI estimator is its simplicity and ease of use, as it requires minimal tuning to achieve exact estimates. Even in scenarios with varying magnitudes of measurement error, the performance of the estimates found using MINMI remained consistent relative to other methods. Since it directly solves the quantile function $\PP_{\theta_q}(S(\bm{W}) > S(\bm{x}))$, estimates found using MINMI have run times that are several orders of magnitude faster than other approaches such as our other proposed method, SI-RM.

However, the MINMI estimator also comes with several caveats. Although it requires little tuning, the choice of the Monte Carlo sample size ($B$) remains a question. Since the Monte Carlo error variation formula (\autoref{theorem: delta-method-variance}) was derived using the delta method, it is only true for large $B$. In our simulation studies and applications, we encountered scenarios where the estimated number of Monte Carlo samples was too small, indicating that the variance found using the delta method was underestimating the true Monte Carlo error variance. In such scenarios, it becomes unclear how to choose $B$, and we suggest a conservative choice of $B = 100$.

Furthermore, since the MINMI estimator exploits uniformity assumptions to estimate extinction times, it is not directly applicable to settings where the uniformity assumption is invalid. In particular, there is evidence to suggest that this assumption is generally not valid, since one can expect the fossils to dwindle near speciation/extinction \cite{Lee2010, Signor1982, WangMarshall2016}.

On the other hand, the main advantage of the SI-RM estimator is its flexibility and ability to accommodate arbitrarily chosen models. Since it uses simulated inversion to produce estimates, the SI-RM estimator can be adapted to fit any model as long as we are able to simulate from it. SI-RM is a method based on the SI estimator proposed by \citet{Huang2019}, applying stochastic approximation in the form of the Robbins-Monro process to strategically select values of $\theta$ rather than defining an arbitrary vector of candidate values to simulate from. This modification results in run times that are significantly faster than SI-UGM, as shown in our simulation studies.

Unfortunately, the flexibility of the proposed SI-RM estimator comes at the cost of additional choices that can significantly affect convergence, such as the step size constant. Although procedures do exist for selecting the step size constant, these rely on some asymptotic normality properties, which should be checked. For our purposes, the heuristic for finding the step length proportionality constant $k$ proposed by \citet{Garthwaite1992} appeared adequate, although alternative approaches to estimating the optimal step length may result in better performance. Moreover, as shown in \autoref{fig:SI-RM-starting-values}, poor starting values can result in slow convergence, as the step length proposed by \citet{Garthwaite1992} is adaptive, using a point estimate as a point of reference for constructing confidence intervals. 

Another choice that can pose a challenge is the choice of stopping criteria. Due to the circular problem outlined in \autoref{subsec:si-rm-stopping-criteria}, where the delta method approximation for estimator variance depends on the gradient of the quantile function at the unknown value $\theta_q$. Although we used the results from the SI-UGM estimator to circumvent this problem, this was largely for illustrative purposes as in practice this becomes redundant. Even when using this workaround, we still observed the Robbins-Monro process stopping before convergence, indicating that the variance approximation from the delta method is being underestimated. Hence, although the flexibility of the SI-RM method is a significant advantage for relaxing certain assumptions, there exists a trade-off in the complexity of these decisions.

The implementation of the SI-RM estimator also relies on an appropriately chosen point estimate $\hat\theta(\bm{w})$, and is not designed to find point estimates. This is because the step lengths are computed as a proportion of the distance between $\hat\theta(\bm{w})$ and the current estimate $\hat\theta_{q; i}$. As a result, the $i$\textsuperscript{th} estimate is ``attracted" to the point estimate, and the distance moved by the Robbins-Monro process decreases as the process approaches $\hat\theta(\bm{w})$. When searching for the point estimator (e.g. by looking for the median $q=0.5$), SI-RM is guaranteed to converge to $\hat\theta(\bm{w})$ and is not informative in an extinction time context.

Both the MINMI and SI-RM estimator assume that measurement errors are independent and identically distributed, which is not necessarily the case. Under the \hyperref[model: measurement-error]{$\varepsilon$-model}, we assume that the measurement errors are i.i.d with constant variance $\sigma^2$. However, in practice, we often have standard errors for each fossil, and it may be possible that measurement errors are correlated across fossils. Extending these estimators to the multivariate case, with or without the independence assumption, is likely improve their performance, but would require some revision of both the $\varepsilon$-model and the theory for MINMI.

\section{Future Work}

A key assumption present throughout the models considered in this thesis is that fossils are uniformly distributed, which can potentially be relaxed. Although it is common practice to assume uniformity, there is substantial evidence to suggest this is not valid as we can expect a species to be less abundant closer to speciation and extinction \cite{Lee2010, WangMarshall2016}. For instance, one approach may be to use recovery rates from other fossil datasets' to estimate the distribution via kernel density estimation, or model fossil recovery as with an exponential distribution or a ``reflected beta distribution" \cite{Bradshaw2012, Wang2016}. As such, a future direction may be to investigate the robustness of the proposed methods to alternate models, or adjust their implementations by relaxing the uniformity assumption in both the \hyperref[model: no-measurement-error]{$\delta$-model} and the \hyperref[model: measurement-error]{$\varepsilon$-model}.

In the same vein, another area for future work is to investigate relaxing the independence of measurement error, which is a common assumption in the literature that has been carried over into this thesis. Since radiocarbon ages are often calibrated using the same calibration curves, it is possible for the measurement errors to be correlated. This is known as a ``curve offset" \cite{Ramsey2010}. In his master's thesis, \citet{King2020} proposed an application of simulated inversion to account for calibration errors, showing some improvement to accuracy and efficiency. Since SI-RM is based on simulated inversion, we expect the adjustments he proposed to work similarly for SI-RM. On the other hand, more work needs to be done to adapt the MINMI estimator to correlated calibration errors.

When applying our methods to datasets of real fossil records, we found that our confidence intervals appeared to narrow in relation to sample size. Although we would expect there to be an inverse relationship between sample size and confidence interval width, we were unable to investigate the relationship between width and sample size in our simulation studies due to time constraints. As a result, it is currently unclear if there is strong evidence for a relationship between confidence interval width/coverage probability and sample size, as we only observed this in our applications of the proposed methods to 4 real datasets. Since sample sizes vary from application to application, understanding this relationship may be a worthwhile venture.

An assumption for all inversion-based estimators is that the quantile function is stochastically increasing in $\theta$, which we were unable to prove this is true for all values of $\theta$ and $\varepsilon$ under the \hyperref[model: measurement-error]{$\varepsilon$-model}. Although we were able to numerically verify this result, this remains an area of investigation for the future that may benefit the development of other estimation methods based on test-statistic inversion.

The performance of estimates and confidence intervals found using test-statistic inversion can vary somewhat based on the choice of the test-statistic. For the MINMI estimator, we used the minimum statistic for inversion under the \hyperref[model: measurement-error]{$\varepsilon$-model}. This was chosen because the minimum statistic is the MLE under the \hyperref[model: no-measurement-error]{$\delta$-model}; however, a better statistic could potentially be found that may further improve estimates. Similarly, a different test statistic may enable better convergence or more optimal step sizes for the SI-RM estimator, since the step size and stopping criteria are influenced by the choice of test-statistic.

For the SI-RM estimator, the choice of step-length constant and stopping criteria are areas of improvement that can potentially improve the performance of the estimator. The choice of step size currently implements the recommendations made by \citet{Garthwaite1992}, who proposed a heuristic that gives asymptotically exact results where the distribution of the point estimate is similar to a normal distribution, which could be further refined: for instance, \citet{LlyodBotev2015} suggested an alternative step-length constant that could be used to refine the step length constant following a burn-in period. Furthermore, additional work needs to be done to identify appropriate stopping criteria for the RM process, as stopping too early results in poor coverage and incorrect estimates. Investigating these areas may lead to improvements in the SI-RM estimator's ease of use.

Finally, although some \texttt{R} code implementing the MINMI and SI-RM estimators has been provided in \autoref{apx:code-minmi} and \autoref{apx:code-si-rm}, these are far from easy to use. Thus, there also remains work to be done in developing software packages for our proposed methods and improve usability for palaeontologists.
