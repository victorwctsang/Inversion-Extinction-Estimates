
Things to cover:

\begin{itemize}
    \item \textbf{What's the problem we're trying to solve?}: estimating confidence intervals for extinction times and dealing with fossil recovery \& measurement error assumptions
    \item \textbf{Why does estimating extinction times matter?}
    \item \textbf{What have other people tried so far?}
    \item \textbf{What do these other methods struggle with?}
        % \begin{itemize}
        %     \item \textbf{What assumptions do/don't they make?}
        %     \item signor lipps effect
        % \end{itemize}
    \item \textbf{What do we propose?}
    \item \textbf{How does our proposition hold up?}
\end{itemize}

% These methods can be divided into "first", "second", and "third" generation methods depending on the assumptions and data used to derive the estimate of extinction times. First generation approaches assume uniform fossil preservation and recovery; second generation approaches assume non-uniform recovery, either inferring or requiring information about recovery potential; and third generation approaches consider external factors to do with stratigraphic and environmental factors to estimate species extinction. For the purposes of this paper, we will focus on first and second generation models as third generation models require large sample sizes and detailed knowledge of stratigraphic/environmental factors, which are often hard to come by.
