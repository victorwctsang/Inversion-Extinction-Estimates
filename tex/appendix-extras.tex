% \begin{figure}[ht]
%     \centering
%     \resizebox{0.45\linewidth}{!}{% Created by tikzDevice version 0.12.3.1 on 2022-10-23 12:31:10
% !TEX encoding = UTF-8 Unicode
\begin{tikzpicture}[x=1pt,y=1pt]
\definecolor{fillColor}{RGB}{255,255,255}
\path[use as bounding box,fill=fillColor,fill opacity=0.00] (0,0) rectangle (505.89,505.89);
\begin{scope}
\path[clip] ( 49.20, 61.20) rectangle (480.69,456.69);
\definecolor{drawColor}{RGB}{0,0,255}

\path[draw=drawColor,line width= 0.4pt,line join=round,line cap=round] ( 69.58,437.96) -- ( 81.49,426.88);

\path[draw=drawColor,line width= 0.4pt,line join=round,line cap=round] ( 91.56,420.84) -- ( 94.90,419.69);

\path[draw=drawColor,line width= 0.4pt,line join=round,line cap=round] (104.96,413.63) -- (107.58,411.19);

\path[draw=drawColor,line width= 0.4pt,line join=round,line cap=round] (116.17,402.81) -- (124.95,393.85);

\path[draw=drawColor,line width= 0.4pt,line join=round,line cap=round] (139.92,379.25) -- (143.42,375.25);

\path[draw=drawColor,line width= 0.4pt,line join=round,line cap=round] (152.01,366.93) -- (152.03,366.91);

\path[draw=drawColor,line width= 0.4pt,line join=round,line cap=round] (161.19,359.16) -- (163.55,357.09);

\path[draw=drawColor,line width= 0.4pt,line join=round,line cap=round] (172.06,348.66) -- (173.39,347.15);

\path[draw=drawColor,line width= 0.4pt,line join=round,line cap=round] (181.63,338.43) -- (183.38,336.67);

\path[draw=drawColor,line width= 0.4pt,line join=round,line cap=round] (191.02,327.49) -- (194.69,322.14);

\path[draw=drawColor,line width= 0.4pt,line join=round,line cap=round] (202.76,313.43) -- (203.66,312.71);

\path[draw=drawColor,line width= 0.4pt,line join=round,line cap=round] (213.33,305.63) -- (214.42,304.90);

\path[draw=drawColor,line width= 0.4pt,line join=round,line cap=round] (233.85,292.13) -- (235.40,290.62);

\path[draw=drawColor,line width= 0.4pt,line join=round,line cap=round] (244.33,282.62) -- (244.94,282.11);

\path[draw=drawColor,line width= 0.4pt,line join=round,line cap=round] (253.32,273.61) -- (255.81,270.50);

\path[draw=drawColor,line width= 0.4pt,line join=round,line cap=round] (263.82,261.59) -- (266.02,259.39);

\path[draw=drawColor,line width= 0.4pt,line join=round,line cap=round] (274.72,251.13) -- (275.83,250.12);

\path[draw=drawColor,line width= 0.4pt,line join=round,line cap=round] (284.91,242.28) -- (285.87,241.49);

\path[draw=drawColor,line width= 0.4pt,line join=round,line cap=round] (295.01,233.72) -- (296.35,232.53);

\path[draw=drawColor,line width= 0.4pt,line join=round,line cap=round] (305.35,224.58) -- (306.62,223.47);

\path[draw=drawColor,line width= 0.4pt,line join=round,line cap=round] (314.76,214.73) -- (317.57,211.04);

\path[draw=drawColor,line width= 0.4pt,line join=round,line cap=round] (326.29,203.08) -- (326.40,203.01);

\path[draw=drawColor,line width= 0.4pt,line join=round,line cap=round] (336.18,196.09) -- (337.04,195.41);

\path[draw=drawColor,line width= 0.4pt,line join=round,line cap=round] (346.77,188.41) -- (346.97,188.28);

\path[draw=drawColor,line width= 0.4pt,line join=round,line cap=round] (356.34,180.86) -- (357.96,179.32);

\path[draw=drawColor,line width= 0.4pt,line join=round,line cap=round] (366.55,170.94) -- (368.27,169.21);

\path[draw=drawColor,line width= 0.4pt,line join=round,line cap=round] (377.00,160.98) -- (378.29,159.83);

\path[draw=drawColor,line width= 0.4pt,line join=round,line cap=round] (387.12,151.70) -- (388.67,150.22);

\path[draw=drawColor,line width= 0.4pt,line join=round,line cap=round] (396.54,141.23) -- (399.70,136.90);

\path[draw=drawColor,line width= 0.4pt,line join=round,line cap=round] (407.39,127.73) -- (409.33,125.72);

\path[draw=drawColor,line width= 0.4pt,line join=round,line cap=round] (417.94,117.37) -- (419.28,116.15);

\path[draw=drawColor,line width= 0.4pt,line join=round,line cap=round] (427.77,107.69) -- (429.93,105.32);

\path[draw=drawColor,line width= 0.4pt,line join=round,line cap=round] (438.71, 97.20) -- (439.48, 96.61);

\path[draw=drawColor,line width= 0.4pt,line join=round,line cap=round] (457.99, 85.94) -- (461.18, 81.57);

\path[draw=drawColor,line width= 0.4pt,line join=round,line cap=round] ( 65.18,442.04) circle (  2.25);

\path[draw=drawColor,line width= 0.4pt,line join=round,line cap=round] ( 85.89,422.80) circle (  2.25);

\path[draw=drawColor,line width= 0.4pt,line join=round,line cap=round] (100.58,417.73) circle (  2.25);

\path[draw=drawColor,line width= 0.4pt,line join=round,line cap=round] (111.97,407.10) circle (  2.25);

\path[draw=drawColor,line width= 0.4pt,line join=round,line cap=round] (129.15,389.57) circle (  2.25);

\path[draw=drawColor,line width= 0.4pt,line join=round,line cap=round] (135.97,383.77) circle (  2.25);

\path[draw=drawColor,line width= 0.4pt,line join=round,line cap=round] (147.37,370.73) circle (  2.25);

\path[draw=drawColor,line width= 0.4pt,line join=round,line cap=round] (156.68,363.11) circle (  2.25);

\path[draw=drawColor,line width= 0.4pt,line join=round,line cap=round] (168.07,353.14) circle (  2.25);

\path[draw=drawColor,line width= 0.4pt,line join=round,line cap=round] (177.38,342.67) circle (  2.25);

\path[draw=drawColor,line width= 0.4pt,line join=round,line cap=round] (187.63,332.43) circle (  2.25);

\path[draw=drawColor,line width= 0.4pt,line join=round,line cap=round] (198.09,317.19) circle (  2.25);

\path[draw=drawColor,line width= 0.4pt,line join=round,line cap=round] (208.33,308.95) circle (  2.25);

\path[draw=drawColor,line width= 0.4pt,line join=round,line cap=round] (219.42,301.57) circle (  2.25);

\path[draw=drawColor,line width= 0.4pt,line join=round,line cap=round] (229.55,296.32) circle (  2.25);

\path[draw=drawColor,line width= 0.4pt,line join=round,line cap=round] (239.70,286.44) circle (  2.25);

\path[draw=drawColor,line width= 0.4pt,line join=round,line cap=round] (249.57,278.29) circle (  2.25);

\path[draw=drawColor,line width= 0.4pt,line join=round,line cap=round] (259.56,265.82) circle (  2.25);

\path[draw=drawColor,line width= 0.4pt,line join=round,line cap=round] (270.27,255.16) circle (  2.25);

\path[draw=drawColor,line width= 0.4pt,line join=round,line cap=round] (280.27,246.08) circle (  2.25);

\path[draw=drawColor,line width= 0.4pt,line join=round,line cap=round] (290.52,237.69) circle (  2.25);

\path[draw=drawColor,line width= 0.4pt,line join=round,line cap=round] (300.85,228.55) circle (  2.25);

\path[draw=drawColor,line width= 0.4pt,line join=round,line cap=round] (311.12,219.50) circle (  2.25);

\path[draw=drawColor,line width= 0.4pt,line join=round,line cap=round] (321.21,206.27) circle (  2.25);

\path[draw=drawColor,line width= 0.4pt,line join=round,line cap=round] (331.48,199.82) circle (  2.25);

\path[draw=drawColor,line width= 0.4pt,line join=round,line cap=round] (341.75,191.68) circle (  2.25);

\path[draw=drawColor,line width= 0.4pt,line join=round,line cap=round] (352.00,185.00) circle (  2.25);

\path[draw=drawColor,line width= 0.4pt,line join=round,line cap=round] (362.30,175.18) circle (  2.25);

\path[draw=drawColor,line width= 0.4pt,line join=round,line cap=round] (372.52,164.97) circle (  2.25);

\path[draw=drawColor,line width= 0.4pt,line join=round,line cap=round] (382.78,155.85) circle (  2.25);

\path[draw=drawColor,line width= 0.4pt,line join=round,line cap=round] (393.01,146.08) circle (  2.25);

\path[draw=drawColor,line width= 0.4pt,line join=round,line cap=round] (403.23,132.05) circle (  2.25);

\path[draw=drawColor,line width= 0.4pt,line join=round,line cap=round] (413.49,121.40) circle (  2.25);

\path[draw=drawColor,line width= 0.4pt,line join=round,line cap=round] (423.73,112.12) circle (  2.25);

\path[draw=drawColor,line width= 0.4pt,line join=round,line cap=round] (433.98,100.88) circle (  2.25);

\path[draw=drawColor,line width= 0.4pt,line join=round,line cap=round] (444.21, 92.93) circle (  2.25);

\path[draw=drawColor,line width= 0.4pt,line join=round,line cap=round] (454.46, 90.79) circle (  2.25);

\path[draw=drawColor,line width= 0.4pt,line join=round,line cap=round] (464.71, 76.72) circle (  2.25);
\end{scope}
\begin{scope}
\path[clip] (  0.00,  0.00) rectangle (505.89,505.89);
\definecolor{drawColor}{RGB}{0,0,0}

\path[draw=drawColor,line width= 0.4pt,line join=round,line cap=round] ( 65.18, 61.20) -- (464.71, 61.20);

\path[draw=drawColor,line width= 0.4pt,line join=round,line cap=round] ( 65.18, 61.20) -- ( 65.18, 55.20);

\path[draw=drawColor,line width= 0.4pt,line join=round,line cap=round] (111.97, 61.20) -- (111.97, 55.20);

\path[draw=drawColor,line width= 0.4pt,line join=round,line cap=round] (147.37, 61.20) -- (147.37, 55.20);

\path[draw=drawColor,line width= 0.4pt,line join=round,line cap=round] (182.76, 61.20) -- (182.76, 55.20);

\path[draw=drawColor,line width= 0.4pt,line join=round,line cap=round] (229.55, 61.20) -- (229.55, 55.20);

\path[draw=drawColor,line width= 0.4pt,line join=round,line cap=round] (264.94, 61.20) -- (264.94, 55.20);

\path[draw=drawColor,line width= 0.4pt,line join=round,line cap=round] (300.34, 61.20) -- (300.34, 55.20);

\path[draw=drawColor,line width= 0.4pt,line join=round,line cap=round] (347.13, 61.20) -- (347.13, 55.20);

\path[draw=drawColor,line width= 0.4pt,line join=round,line cap=round] (382.52, 61.20) -- (382.52, 55.20);

\path[draw=drawColor,line width= 0.4pt,line join=round,line cap=round] (417.92, 61.20) -- (417.92, 55.20);

\path[draw=drawColor,line width= 0.4pt,line join=round,line cap=round] (464.71, 61.20) -- (464.71, 55.20);

\node[text=drawColor,anchor=base,inner sep=0pt, outer sep=0pt, scale=  1.00] at ( 65.18, 39.60) {2};

\node[text=drawColor,anchor=base,inner sep=0pt, outer sep=0pt, scale=  1.00] at (111.97, 39.60) {5};

\node[text=drawColor,anchor=base,inner sep=0pt, outer sep=0pt, scale=  1.00] at (147.37, 39.60) {10};

\node[text=drawColor,anchor=base,inner sep=0pt, outer sep=0pt, scale=  1.00] at (182.76, 39.60) {20};

\node[text=drawColor,anchor=base,inner sep=0pt, outer sep=0pt, scale=  1.00] at (229.55, 39.60) {50};

\node[text=drawColor,anchor=base,inner sep=0pt, outer sep=0pt, scale=  1.00] at (264.94, 39.60) {100};

\node[text=drawColor,anchor=base,inner sep=0pt, outer sep=0pt, scale=  1.00] at (300.34, 39.60) {200};

\node[text=drawColor,anchor=base,inner sep=0pt, outer sep=0pt, scale=  1.00] at (347.13, 39.60) {500};

\node[text=drawColor,anchor=base,inner sep=0pt, outer sep=0pt, scale=  1.00] at (382.52, 39.60) {1000};

\node[text=drawColor,anchor=base,inner sep=0pt, outer sep=0pt, scale=  1.00] at (417.92, 39.60) {2000};

\node[text=drawColor,anchor=base,inner sep=0pt, outer sep=0pt, scale=  1.00] at (464.71, 39.60) {5000};

\path[draw=drawColor,line width= 0.4pt,line join=round,line cap=round] ( 49.20, 90.55) -- ( 49.20,415.62);

\path[draw=drawColor,line width= 0.4pt,line join=round,line cap=round] ( 49.20, 90.55) -- ( 43.20, 90.55);

\path[draw=drawColor,line width= 0.4pt,line join=round,line cap=round] ( 49.20,133.67) -- ( 43.20,133.67);

\path[draw=drawColor,line width= 0.4pt,line join=round,line cap=round] ( 49.20,166.29) -- ( 43.20,166.29);

\path[draw=drawColor,line width= 0.4pt,line join=round,line cap=round] ( 49.20,198.91) -- ( 43.20,198.91);

\path[draw=drawColor,line width= 0.4pt,line join=round,line cap=round] ( 49.20,242.03) -- ( 43.20,242.03);

\path[draw=drawColor,line width= 0.4pt,line join=round,line cap=round] ( 49.20,274.64) -- ( 43.20,274.64);

\path[draw=drawColor,line width= 0.4pt,line join=round,line cap=round] ( 49.20,307.26) -- ( 43.20,307.26);

\path[draw=drawColor,line width= 0.4pt,line join=round,line cap=round] ( 49.20,350.38) -- ( 43.20,350.38);

\path[draw=drawColor,line width= 0.4pt,line join=round,line cap=round] ( 49.20,383.00) -- ( 43.20,383.00);

\path[draw=drawColor,line width= 0.4pt,line join=round,line cap=round] ( 49.20,415.62) -- ( 43.20,415.62);

\node[text=drawColor,rotate= 90.00,anchor=base,inner sep=0pt, outer sep=0pt, scale=  1.00] at ( 34.80, 90.55) {20};

\node[text=drawColor,rotate= 90.00,anchor=base,inner sep=0pt, outer sep=0pt, scale=  1.00] at ( 34.80,133.67) {50};

\node[text=drawColor,rotate= 90.00,anchor=base,inner sep=0pt, outer sep=0pt, scale=  1.00] at ( 34.80,166.29) {100};

\node[text=drawColor,rotate= 90.00,anchor=base,inner sep=0pt, outer sep=0pt, scale=  1.00] at ( 34.80,198.91) {200};

\node[text=drawColor,rotate= 90.00,anchor=base,inner sep=0pt, outer sep=0pt, scale=  1.00] at ( 34.80,242.03) {500};

\node[text=drawColor,rotate= 90.00,anchor=base,inner sep=0pt, outer sep=0pt, scale=  1.00] at ( 34.80,274.64) {1000};

\node[text=drawColor,rotate= 90.00,anchor=base,inner sep=0pt, outer sep=0pt, scale=  1.00] at ( 34.80,307.26) {2000};

\node[text=drawColor,rotate= 90.00,anchor=base,inner sep=0pt, outer sep=0pt, scale=  1.00] at ( 34.80,350.38) {5000};

\node[text=drawColor,rotate= 90.00,anchor=base,inner sep=0pt, outer sep=0pt, scale=  1.00] at ( 34.80,383.00) {10000};

\path[draw=drawColor,line width= 0.4pt,line join=round,line cap=round] ( 49.20, 61.20) --
	(480.69, 61.20) --
	(480.69,456.69) --
	( 49.20,456.69) --
	cycle;
\end{scope}
\begin{scope}
\path[clip] (  0.00,  0.00) rectangle (505.89,505.89);
\definecolor{drawColor}{RGB}{0,0,0}

\node[text=drawColor,anchor=base,inner sep=0pt, outer sep=0pt, scale=  1.20] at (264.94,477.15) {\bfseries Sample Var vs. Delta Method (q=0.025)};

\node[text=drawColor,anchor=base,inner sep=0pt, outer sep=0pt, scale=  1.00] at (264.94, 15.60) {B};
\end{scope}
\begin{scope}
\path[clip] ( 49.20, 61.20) rectangle (480.69,456.69);
\definecolor{drawColor}{RGB}{255,0,0}

\path[draw=drawColor,line width= 0.4pt,line join=round,line cap=round] ( 69.51,412.63) -- ( 81.56,401.07);

\path[draw=drawColor,line width= 0.4pt,line join=round,line cap=round] ( 88.36,391.45) -- ( 98.11,369.86);

\path[draw=drawColor,line width= 0.4pt,line join=round,line cap=round] (114.52,366.36) -- (126.60,392.05);

\path[draw=drawColor,line width= 0.4pt,line join=round,line cap=round] (132.50,392.50) -- (132.62,392.33);

\path[draw=drawColor,line width= 0.4pt,line join=round,line cap=round] (139.06,382.20) -- (144.28,373.52);

\path[draw=drawColor,line width= 0.4pt,line join=round,line cap=round] (160.50,358.36) -- (164.25,353.82);

\path[draw=drawColor,line width= 0.4pt,line join=round,line cap=round] (171.21,344.08) -- (174.24,339.16);

\path[draw=drawColor,line width= 0.4pt,line join=round,line cap=round] (180.51,328.93) -- (184.50,322.42);

\path[draw=drawColor,line width= 0.4pt,line join=round,line cap=round] (202.57,314.00) -- (203.85,312.86);

\path[draw=drawColor,line width= 0.4pt,line join=round,line cap=round] (211.77,303.96) -- (215.97,297.97);

\path[draw=drawColor,line width= 0.4pt,line join=round,line cap=round] (223.75,288.90) -- (225.22,287.48);

\path[draw=drawColor,line width= 0.4pt,line join=round,line cap=round] (234.16,279.49) -- (235.10,278.70);

\path[draw=drawColor,line width= 0.4pt,line join=round,line cap=round] (243.71,270.39) -- (245.56,268.32);

\path[draw=drawColor,line width= 0.4pt,line join=round,line cap=round] (264.71,256.55) -- (265.13,256.29);

\path[draw=drawColor,line width= 0.4pt,line join=round,line cap=round] (273.94,248.45) -- (276.60,245.01);

\path[draw=drawColor,line width= 0.4pt,line join=round,line cap=round] (295.43,230.67) -- (295.94,230.31);

\path[draw=drawColor,line width= 0.4pt,line join=round,line cap=round] (305.43,223.00) -- (306.53,222.06);

\path[draw=drawColor,line width= 0.4pt,line join=round,line cap=round] (315.82,214.46) -- (316.52,213.90);

\path[draw=drawColor,line width= 0.4pt,line join=round,line cap=round] (325.40,205.87) -- (327.30,203.91);

\path[draw=drawColor,line width= 0.4pt,line join=round,line cap=round] (335.89,195.54) -- (337.34,194.21);

\path[draw=drawColor,line width= 0.4pt,line join=round,line cap=round] (345.64,185.58) -- (348.10,182.70);

\path[draw=drawColor,line width= 0.4pt,line join=round,line cap=round] (356.21,173.86) -- (358.09,171.95);

\path[draw=drawColor,line width= 0.4pt,line join=round,line cap=round] (366.81,163.71) -- (368.01,162.66);

\path[draw=drawColor,line width= 0.4pt,line join=round,line cap=round] (376.97,154.69) -- (378.32,153.48);

\path[draw=drawColor,line width= 0.4pt,line join=round,line cap=round] (387.38,145.62) -- (388.41,144.75);

\path[draw=drawColor,line width= 0.4pt,line join=round,line cap=round] (397.38,136.79) -- (398.86,135.40);

\path[draw=drawColor,line width= 0.4pt,line join=round,line cap=round] (407.80,127.40) -- (408.93,126.43);

\path[draw=drawColor,line width= 0.4pt,line join=round,line cap=round] (417.93,118.50) -- (419.29,117.26);

\path[draw=drawColor,line width= 0.4pt,line join=round,line cap=round] (428.18,109.20) -- (429.52,107.99);

\path[draw=drawColor,line width= 0.4pt,line join=round,line cap=round] (438.51,100.04) -- (439.68, 99.02);

\path[draw=drawColor,line width= 0.4pt,line join=round,line cap=round] (448.43, 90.82) -- (450.25, 88.98);

\path[draw=drawColor,line width= 0.4pt,line join=round,line cap=round] (459.00, 80.78) -- (460.17, 79.77);

\path[draw=drawColor,line width= 0.4pt,line join=round,line cap=round] ( 65.18,416.78) circle (  2.25);

\path[draw=drawColor,line width= 0.4pt,line join=round,line cap=round] ( 85.89,396.92) circle (  2.25);

\path[draw=drawColor,line width= 0.4pt,line join=round,line cap=round] (100.58,364.39) circle (  2.25);

\path[draw=drawColor,line width= 0.4pt,line join=round,line cap=round] (111.97,360.93) circle (  2.25);

\path[draw=drawColor,line width= 0.4pt,line join=round,line cap=round] (129.15,397.48) circle (  2.25);

\path[draw=drawColor,line width= 0.4pt,line join=round,line cap=round] (135.97,387.35) circle (  2.25);

\path[draw=drawColor,line width= 0.4pt,line join=round,line cap=round] (147.37,368.37) circle (  2.25);

\path[draw=drawColor,line width= 0.4pt,line join=round,line cap=round] (156.68,362.98) circle (  2.25);

\path[draw=drawColor,line width= 0.4pt,line join=round,line cap=round] (168.07,349.20) circle (  2.25);

\path[draw=drawColor,line width= 0.4pt,line join=round,line cap=round] (177.38,334.05) circle (  2.25);

\path[draw=drawColor,line width= 0.4pt,line join=round,line cap=round] (187.63,317.30) circle (  2.25);

\path[draw=drawColor,line width= 0.4pt,line join=round,line cap=round] (198.09,317.98) circle (  2.25);

\path[draw=drawColor,line width= 0.4pt,line join=round,line cap=round] (208.33,308.88) circle (  2.25);

\path[draw=drawColor,line width= 0.4pt,line join=round,line cap=round] (219.42,293.05) circle (  2.25);

\path[draw=drawColor,line width= 0.4pt,line join=round,line cap=round] (229.55,283.33) circle (  2.25);

\path[draw=drawColor,line width= 0.4pt,line join=round,line cap=round] (239.70,274.86) circle (  2.25);

\path[draw=drawColor,line width= 0.4pt,line join=round,line cap=round] (249.57,263.85) circle (  2.25);

\path[draw=drawColor,line width= 0.4pt,line join=round,line cap=round] (259.56,259.65) circle (  2.25);

\path[draw=drawColor,line width= 0.4pt,line join=round,line cap=round] (270.27,253.20) circle (  2.25);

\path[draw=drawColor,line width= 0.4pt,line join=round,line cap=round] (280.27,240.26) circle (  2.25);

\path[draw=drawColor,line width= 0.4pt,line join=round,line cap=round] (290.52,234.11) circle (  2.25);

\path[draw=drawColor,line width= 0.4pt,line join=round,line cap=round] (300.85,226.87) circle (  2.25);

\path[draw=drawColor,line width= 0.4pt,line join=round,line cap=round] (311.12,218.19) circle (  2.25);

\path[draw=drawColor,line width= 0.4pt,line join=round,line cap=round] (321.21,210.17) circle (  2.25);

\path[draw=drawColor,line width= 0.4pt,line join=round,line cap=round] (331.48,199.61) circle (  2.25);

\path[draw=drawColor,line width= 0.4pt,line join=round,line cap=round] (341.75,190.15) circle (  2.25);

\path[draw=drawColor,line width= 0.4pt,line join=round,line cap=round] (352.00,178.14) circle (  2.25);

\path[draw=drawColor,line width= 0.4pt,line join=round,line cap=round] (362.30,167.67) circle (  2.25);

\path[draw=drawColor,line width= 0.4pt,line join=round,line cap=round] (372.52,158.70) circle (  2.25);

\path[draw=drawColor,line width= 0.4pt,line join=round,line cap=round] (382.78,149.47) circle (  2.25);

\path[draw=drawColor,line width= 0.4pt,line join=round,line cap=round] (393.01,140.91) circle (  2.25);

\path[draw=drawColor,line width= 0.4pt,line join=round,line cap=round] (403.23,131.29) circle (  2.25);

\path[draw=drawColor,line width= 0.4pt,line join=round,line cap=round] (413.49,122.54) circle (  2.25);

\path[draw=drawColor,line width= 0.4pt,line join=round,line cap=round] (423.73,113.22) circle (  2.25);

\path[draw=drawColor,line width= 0.4pt,line join=round,line cap=round] (433.98,103.97) circle (  2.25);

\path[draw=drawColor,line width= 0.4pt,line join=round,line cap=round] (444.21, 95.09) circle (  2.25);

\path[draw=drawColor,line width= 0.4pt,line join=round,line cap=round] (454.46, 84.70) circle (  2.25);

\path[draw=drawColor,line width= 0.4pt,line join=round,line cap=round] (464.71, 75.85) circle (  2.25);
\definecolor{drawColor}{RGB}{0,0,0}

\path[draw=drawColor,line width= 0.4pt,line join=round,line cap=round] (395.89,456.69) rectangle (480.69,420.69);
\definecolor{drawColor}{RGB}{0,0,255}

\path[draw=drawColor,line width= 0.4pt,line join=round,line cap=round] (398.59,444.69) -- (416.59,444.69);
\definecolor{drawColor}{RGB}{255,0,0}

\path[draw=drawColor,line width= 0.4pt,line join=round,line cap=round] (398.59,432.69) -- (416.59,432.69);
\definecolor{drawColor}{RGB}{0,0,255}

\path[draw=drawColor,line width= 0.4pt,line join=round,line cap=round] (407.59,444.69) circle (  2.25);
\definecolor{drawColor}{RGB}{255,0,0}

\path[draw=drawColor,line width= 0.4pt,line join=round,line cap=round] (407.59,432.69) circle (  2.25);
\definecolor{drawColor}{RGB}{0,0,0}

\node[text=drawColor,anchor=base west,inner sep=0pt, outer sep=0pt, scale=  1.00] at (425.59,441.25) {Sample Var};

\node[text=drawColor,anchor=base west,inner sep=0pt, outer sep=0pt, scale=  1.00] at (425.59,429.25) {Asymptotic};
\end{scope}
\end{tikzpicture}
}
%     \resizebox{0.45\linewidth}{!}{% Created by tikzDevice version 0.12.3.1 on 2022-10-23 12:41:59
% !TEX encoding = UTF-8 Unicode
\begin{tikzpicture}[x=1pt,y=1pt]
\definecolor{fillColor}{RGB}{255,255,255}
\path[use as bounding box,fill=fillColor,fill opacity=0.00] (0,0) rectangle (505.89,505.89);
\begin{scope}
\path[clip] ( 49.20, 61.20) rectangle (480.69,456.69);
\definecolor{drawColor}{RGB}{0,0,255}

\path[draw=drawColor,line width= 0.4pt,line join=round,line cap=round] ( 69.28,437.66) -- ( 81.79,424.26);

\path[draw=drawColor,line width= 0.4pt,line join=round,line cap=round] ( 91.30,417.28) -- ( 95.16,415.43);

\path[draw=drawColor,line width= 0.4pt,line join=round,line cap=round] (105.05,408.85) -- (107.49,406.67);

\path[draw=drawColor,line width= 0.4pt,line join=round,line cap=round] (115.92,398.16) -- (125.20,387.54);

\path[draw=drawColor,line width= 0.4pt,line join=round,line cap=round] (139.79,374.08) -- (143.55,369.53);

\path[draw=drawColor,line width= 0.4pt,line join=round,line cap=round] (161.20,353.80) -- (163.54,351.77);

\path[draw=drawColor,line width= 0.4pt,line join=round,line cap=round] (172.10,343.39) -- (173.35,342.02);

\path[draw=drawColor,line width= 0.4pt,line join=round,line cap=round] (181.30,333.04) -- (183.71,330.25);

\path[draw=drawColor,line width= 0.4pt,line join=round,line cap=round] (191.31,320.97) -- (194.40,317.00);

\path[draw=drawColor,line width= 0.4pt,line join=round,line cap=round] (202.00,307.72) -- (204.42,304.92);

\path[draw=drawColor,line width= 0.4pt,line join=round,line cap=round] (213.44,297.22) -- (214.31,296.67);

\path[draw=drawColor,line width= 0.4pt,line join=round,line cap=round] (233.73,284.16) -- (235.53,282.31);

\path[draw=drawColor,line width= 0.4pt,line join=round,line cap=round] (244.10,273.91) -- (245.18,272.91);

\path[draw=drawColor,line width= 0.4pt,line join=round,line cap=round] (253.21,264.05) -- (255.92,260.50);

\path[draw=drawColor,line width= 0.4pt,line join=round,line cap=round] (263.51,251.22) -- (266.33,247.98);

\path[draw=drawColor,line width= 0.4pt,line join=round,line cap=round] (275.20,240.04) -- (275.34,239.94);

\path[draw=drawColor,line width= 0.4pt,line join=round,line cap=round] (285.07,232.92) -- (285.72,232.43);

\path[draw=drawColor,line width= 0.4pt,line join=round,line cap=round] (294.97,224.81) -- (296.39,223.53);

\path[draw=drawColor,line width= 0.4pt,line join=round,line cap=round] (305.42,215.63) -- (306.54,214.67);

\path[draw=drawColor,line width= 0.4pt,line join=round,line cap=round] (315.00,206.21) -- (317.33,203.47);

\path[draw=drawColor,line width= 0.4pt,line join=round,line cap=round] (336.55,191.02) -- (336.68,190.94);

\path[draw=drawColor,line width= 0.4pt,line join=round,line cap=round] (356.38,179.11) -- (357.92,177.68);

\path[draw=drawColor,line width= 0.4pt,line join=round,line cap=round] (366.95,169.80) -- (367.86,169.06);

\path[draw=drawColor,line width= 0.4pt,line join=round,line cap=round] (377.31,161.67) -- (377.98,161.17);

\path[draw=drawColor,line width= 0.4pt,line join=round,line cap=round] (387.19,153.50) -- (388.60,152.21);

\path[draw=drawColor,line width= 0.4pt,line join=round,line cap=round] (396.67,143.39) -- (399.57,139.62);

\path[draw=drawColor,line width= 0.4pt,line join=round,line cap=round] (407.43,130.58) -- (409.29,128.68);

\path[draw=drawColor,line width= 0.4pt,line join=round,line cap=round] (417.71,120.13) -- (419.51,118.30);

\path[draw=drawColor,line width= 0.4pt,line join=round,line cap=round] (427.84,109.66) -- (429.87,107.50);

\path[draw=drawColor,line width= 0.4pt,line join=round,line cap=round] (438.52, 99.22) -- (439.67, 98.24);

\path[draw=drawColor,line width= 0.4pt,line join=round,line cap=round] (458.75, 86.61) -- (460.43, 84.96);

\path[draw=drawColor,line width= 0.4pt,line join=round,line cap=round] ( 65.18,442.04) circle (  2.25);

\path[draw=drawColor,line width= 0.4pt,line join=round,line cap=round] ( 85.89,419.88) circle (  2.25);

\path[draw=drawColor,line width= 0.4pt,line join=round,line cap=round] (100.58,412.84) circle (  2.25);

\path[draw=drawColor,line width= 0.4pt,line join=round,line cap=round] (111.97,402.68) circle (  2.25);

\path[draw=drawColor,line width= 0.4pt,line join=round,line cap=round] (129.15,383.02) circle (  2.25);

\path[draw=drawColor,line width= 0.4pt,line join=round,line cap=round] (135.97,378.71) circle (  2.25);

\path[draw=drawColor,line width= 0.4pt,line join=round,line cap=round] (147.37,364.90) circle (  2.25);

\path[draw=drawColor,line width= 0.4pt,line join=round,line cap=round] (156.68,357.74) circle (  2.25);

\path[draw=drawColor,line width= 0.4pt,line join=round,line cap=round] (168.07,347.83) circle (  2.25);

\path[draw=drawColor,line width= 0.4pt,line join=round,line cap=round] (177.38,337.58) circle (  2.25);

\path[draw=drawColor,line width= 0.4pt,line join=round,line cap=round] (187.63,325.71) circle (  2.25);

\path[draw=drawColor,line width= 0.4pt,line join=round,line cap=round] (198.09,312.27) circle (  2.25);

\path[draw=drawColor,line width= 0.4pt,line join=round,line cap=round] (208.33,300.37) circle (  2.25);

\path[draw=drawColor,line width= 0.4pt,line join=round,line cap=round] (219.42,293.52) circle (  2.25);

\path[draw=drawColor,line width= 0.4pt,line join=round,line cap=round] (229.55,288.47) circle (  2.25);

\path[draw=drawColor,line width= 0.4pt,line join=round,line cap=round] (239.70,278.00) circle (  2.25);

\path[draw=drawColor,line width= 0.4pt,line join=round,line cap=round] (249.57,268.82) circle (  2.25);

\path[draw=drawColor,line width= 0.4pt,line join=round,line cap=round] (259.56,255.74) circle (  2.25);

\path[draw=drawColor,line width= 0.4pt,line join=round,line cap=round] (270.27,243.46) circle (  2.25);

\path[draw=drawColor,line width= 0.4pt,line join=round,line cap=round] (280.27,236.52) circle (  2.25);

\path[draw=drawColor,line width= 0.4pt,line join=round,line cap=round] (290.52,228.83) circle (  2.25);

\path[draw=drawColor,line width= 0.4pt,line join=round,line cap=round] (300.85,219.51) circle (  2.25);

\path[draw=drawColor,line width= 0.4pt,line join=round,line cap=round] (311.12,210.79) circle (  2.25);

\path[draw=drawColor,line width= 0.4pt,line join=round,line cap=round] (321.21,198.90) circle (  2.25);

\path[draw=drawColor,line width= 0.4pt,line join=round,line cap=round] (331.48,194.23) circle (  2.25);

\path[draw=drawColor,line width= 0.4pt,line join=round,line cap=round] (341.75,187.73) circle (  2.25);

\path[draw=drawColor,line width= 0.4pt,line join=round,line cap=round] (352.00,183.21) circle (  2.25);

\path[draw=drawColor,line width= 0.4pt,line join=round,line cap=round] (362.30,173.59) circle (  2.25);

\path[draw=drawColor,line width= 0.4pt,line join=round,line cap=round] (372.52,165.27) circle (  2.25);

\path[draw=drawColor,line width= 0.4pt,line join=round,line cap=round] (382.78,157.56) circle (  2.25);

\path[draw=drawColor,line width= 0.4pt,line join=round,line cap=round] (393.01,148.15) circle (  2.25);

\path[draw=drawColor,line width= 0.4pt,line join=round,line cap=round] (403.23,134.86) circle (  2.25);

\path[draw=drawColor,line width= 0.4pt,line join=round,line cap=round] (413.49,124.40) circle (  2.25);

\path[draw=drawColor,line width= 0.4pt,line join=round,line cap=round] (423.73,114.03) circle (  2.25);

\path[draw=drawColor,line width= 0.4pt,line join=round,line cap=round] (433.98,103.13) circle (  2.25);

\path[draw=drawColor,line width= 0.4pt,line join=round,line cap=round] (444.21, 94.32) circle (  2.25);

\path[draw=drawColor,line width= 0.4pt,line join=round,line cap=round] (454.46, 90.81) circle (  2.25);

\path[draw=drawColor,line width= 0.4pt,line join=round,line cap=round] (464.71, 80.76) circle (  2.25);
\end{scope}
\begin{scope}
\path[clip] (  0.00,  0.00) rectangle (505.89,505.89);
\definecolor{drawColor}{RGB}{0,0,0}

\path[draw=drawColor,line width= 0.4pt,line join=round,line cap=round] ( 65.18, 61.20) -- (464.71, 61.20);

\path[draw=drawColor,line width= 0.4pt,line join=round,line cap=round] ( 65.18, 61.20) -- ( 65.18, 55.20);

\path[draw=drawColor,line width= 0.4pt,line join=round,line cap=round] (111.97, 61.20) -- (111.97, 55.20);

\path[draw=drawColor,line width= 0.4pt,line join=round,line cap=round] (147.37, 61.20) -- (147.37, 55.20);

\path[draw=drawColor,line width= 0.4pt,line join=round,line cap=round] (182.76, 61.20) -- (182.76, 55.20);

\path[draw=drawColor,line width= 0.4pt,line join=round,line cap=round] (229.55, 61.20) -- (229.55, 55.20);

\path[draw=drawColor,line width= 0.4pt,line join=round,line cap=round] (264.94, 61.20) -- (264.94, 55.20);

\path[draw=drawColor,line width= 0.4pt,line join=round,line cap=round] (300.34, 61.20) -- (300.34, 55.20);

\path[draw=drawColor,line width= 0.4pt,line join=round,line cap=round] (347.13, 61.20) -- (347.13, 55.20);

\path[draw=drawColor,line width= 0.4pt,line join=round,line cap=round] (382.52, 61.20) -- (382.52, 55.20);

\path[draw=drawColor,line width= 0.4pt,line join=round,line cap=round] (417.92, 61.20) -- (417.92, 55.20);

\path[draw=drawColor,line width= 0.4pt,line join=round,line cap=round] (464.71, 61.20) -- (464.71, 55.20);

\node[text=drawColor,anchor=base,inner sep=0pt, outer sep=0pt, scale=  1.00] at ( 65.18, 39.60) {2};

\node[text=drawColor,anchor=base,inner sep=0pt, outer sep=0pt, scale=  1.00] at (111.97, 39.60) {5};

\node[text=drawColor,anchor=base,inner sep=0pt, outer sep=0pt, scale=  1.00] at (147.37, 39.60) {10};

\node[text=drawColor,anchor=base,inner sep=0pt, outer sep=0pt, scale=  1.00] at (182.76, 39.60) {20};

\node[text=drawColor,anchor=base,inner sep=0pt, outer sep=0pt, scale=  1.00] at (229.55, 39.60) {50};

\node[text=drawColor,anchor=base,inner sep=0pt, outer sep=0pt, scale=  1.00] at (264.94, 39.60) {100};

\node[text=drawColor,anchor=base,inner sep=0pt, outer sep=0pt, scale=  1.00] at (300.34, 39.60) {200};

\node[text=drawColor,anchor=base,inner sep=0pt, outer sep=0pt, scale=  1.00] at (347.13, 39.60) {500};

\node[text=drawColor,anchor=base,inner sep=0pt, outer sep=0pt, scale=  1.00] at (382.52, 39.60) {1000};

\node[text=drawColor,anchor=base,inner sep=0pt, outer sep=0pt, scale=  1.00] at (417.92, 39.60) {2000};

\node[text=drawColor,anchor=base,inner sep=0pt, outer sep=0pt, scale=  1.00] at (464.71, 39.60) {5000};

\path[draw=drawColor,line width= 0.4pt,line join=round,line cap=round] ( 49.20,114.19) -- ( 49.20,455.01);

\path[draw=drawColor,line width= 0.4pt,line join=round,line cap=round] ( 49.20,114.19) -- ( 43.20,114.19);

\path[draw=drawColor,line width= 0.4pt,line join=round,line cap=round] ( 49.20,145.27) -- ( 43.20,145.27);

\path[draw=drawColor,line width= 0.4pt,line join=round,line cap=round] ( 49.20,217.44) -- ( 43.20,217.44);

\path[draw=drawColor,line width= 0.4pt,line join=round,line cap=round] ( 49.20,248.52) -- ( 43.20,248.52);

\path[draw=drawColor,line width= 0.4pt,line join=round,line cap=round] ( 49.20,320.69) -- ( 43.20,320.69);

\path[draw=drawColor,line width= 0.4pt,line join=round,line cap=round] ( 49.20,351.77) -- ( 43.20,351.77);

\path[draw=drawColor,line width= 0.4pt,line join=round,line cap=round] ( 49.20,423.93) -- ( 43.20,423.93);

\path[draw=drawColor,line width= 0.4pt,line join=round,line cap=round] ( 49.20,455.01) -- ( 43.20,455.01);

\node[text=drawColor,rotate= 90.00,anchor=base,inner sep=0pt, outer sep=0pt, scale=  1.00] at ( 34.80,114.19) {5};

\node[text=drawColor,rotate= 90.00,anchor=base,inner sep=0pt, outer sep=0pt, scale=  1.00] at ( 34.80,145.27) {10};

\node[text=drawColor,rotate= 90.00,anchor=base,inner sep=0pt, outer sep=0pt, scale=  1.00] at ( 34.80,217.44) {50};

\node[text=drawColor,rotate= 90.00,anchor=base,inner sep=0pt, outer sep=0pt, scale=  1.00] at ( 34.80,248.52) {100};

\node[text=drawColor,rotate= 90.00,anchor=base,inner sep=0pt, outer sep=0pt, scale=  1.00] at ( 34.80,320.69) {500};

\node[text=drawColor,rotate= 90.00,anchor=base,inner sep=0pt, outer sep=0pt, scale=  1.00] at ( 34.80,351.77) {1000};

\node[text=drawColor,rotate= 90.00,anchor=base,inner sep=0pt, outer sep=0pt, scale=  1.00] at ( 34.80,423.93) {5000};

\node[text=drawColor,rotate= 90.00,anchor=base,inner sep=0pt, outer sep=0pt, scale=  1.00] at ( 34.80,455.01) {10000};

\path[draw=drawColor,line width= 0.4pt,line join=round,line cap=round] ( 49.20, 61.20) --
	(480.69, 61.20) --
	(480.69,456.69) --
	( 49.20,456.69) --
	cycle;
\end{scope}
\begin{scope}
\path[clip] (  0.00,  0.00) rectangle (505.89,505.89);
\definecolor{drawColor}{RGB}{0,0,0}

\node[text=drawColor,anchor=base,inner sep=0pt, outer sep=0pt, scale=  1.20] at (264.94,477.15) {\bfseries Sample Var vs. Asymptotic Var (q=0.975)};

\node[text=drawColor,anchor=base,inner sep=0pt, outer sep=0pt, scale=  1.00] at (264.94, 15.60) {B};
\end{scope}
\begin{scope}
\path[clip] ( 49.20, 61.20) rectangle (480.69,456.69);
\definecolor{drawColor}{RGB}{255,0,0}

\path[draw=drawColor,line width= 0.4pt,line join=round,line cap=round] ( 71.14,312.44) -- ( 79.93,311.35);

\path[draw=drawColor,line width= 0.4pt,line join=round,line cap=round] ( 88.50,305.22) -- ( 97.96,285.73);

\path[draw=drawColor,line width= 0.4pt,line join=round,line cap=round] (105.59,277.03) -- (106.96,276.13);

\path[draw=drawColor,line width= 0.4pt,line join=round,line cap=round] (112.99,278.74) -- (128.13,366.74);

\path[draw=drawColor,line width= 0.4pt,line join=round,line cap=round] (132.18,367.47) -- (132.95,366.15);

\path[draw=drawColor,line width= 0.4pt,line join=round,line cap=round] (138.90,355.74) -- (144.43,345.87);

\path[draw=drawColor,line width= 0.4pt,line join=round,line cap=round] (160.52,335.93) -- (164.23,331.49);

\path[draw=drawColor,line width= 0.4pt,line join=round,line cap=round] (171.37,321.88) -- (174.08,317.76);

\path[draw=drawColor,line width= 0.4pt,line join=round,line cap=round] (180.82,307.84) -- (184.19,303.01);

\path[draw=drawColor,line width= 0.4pt,line join=round,line cap=round] (202.28,288.67) -- (204.14,286.76);

\path[draw=drawColor,line width= 0.4pt,line join=round,line cap=round] (212.03,277.74) -- (215.72,273.01);

\path[draw=drawColor,line width= 0.4pt,line join=round,line cap=round] (222.82,263.35) -- (226.14,258.54);

\path[draw=drawColor,line width= 0.4pt,line join=round,line cap=round] (244.58,255.65) -- (244.69,255.58);

\path[draw=drawColor,line width= 0.4pt,line join=round,line cap=round] (274.07,240.60) -- (276.47,237.67);

\path[draw=drawColor,line width= 0.4pt,line join=round,line cap=round] (284.41,228.69) -- (286.37,226.64);

\path[draw=drawColor,line width= 0.4pt,line join=round,line cap=round] (294.86,218.16) -- (296.50,216.60);

\path[draw=drawColor,line width= 0.4pt,line join=round,line cap=round] (325.42,196.20) -- (327.28,194.31);

\path[draw=drawColor,line width= 0.4pt,line join=round,line cap=round] (345.49,179.32) -- (348.25,175.86);

\path[draw=drawColor,line width= 0.4pt,line join=round,line cap=round] (356.50,167.20) -- (357.80,166.05);

\path[draw=drawColor,line width= 0.4pt,line join=round,line cap=round] (367.02,158.38) -- (367.80,157.77);

\path[draw=drawColor,line width= 0.4pt,line join=round,line cap=round] (376.87,149.94) -- (378.43,148.46);

\path[draw=drawColor,line width= 0.4pt,line join=round,line cap=round] (387.52,140.65) -- (388.28,140.06);

\path[draw=drawColor,line width= 0.4pt,line join=round,line cap=round] (397.47,132.36) -- (398.77,131.19);

\path[draw=drawColor,line width= 0.4pt,line join=round,line cap=round] (408.00,123.54) -- (408.73,122.98);

\path[draw=drawColor,line width= 0.4pt,line join=round,line cap=round] (418.18,115.58) -- (419.05,114.89);

\path[draw=drawColor,line width= 0.4pt,line join=round,line cap=round] (428.49,107.49) -- (429.21,106.94);

\path[draw=drawColor,line width= 0.4pt,line join=round,line cap=round] (438.52, 99.37) -- (439.67, 98.37);

\path[draw=drawColor,line width= 0.4pt,line join=round,line cap=round] (448.44, 90.19) -- (450.24, 88.37);

\path[draw=drawColor,line width= 0.4pt,line join=round,line cap=round] (459.13, 80.34) -- (460.04, 79.61);

\path[draw=drawColor,line width= 0.4pt,line join=round,line cap=round] ( 65.18,313.18) circle (  2.25);

\path[draw=drawColor,line width= 0.4pt,line join=round,line cap=round] ( 85.89,310.62) circle (  2.25);

\path[draw=drawColor,line width= 0.4pt,line join=round,line cap=round] (100.58,280.33) circle (  2.25);

\path[draw=drawColor,line width= 0.4pt,line join=round,line cap=round] (111.97,272.83) circle (  2.25);

\path[draw=drawColor,line width= 0.4pt,line join=round,line cap=round] (129.15,372.65) circle (  2.25);

\path[draw=drawColor,line width= 0.4pt,line join=round,line cap=round] (135.97,360.97) circle (  2.25);

\path[draw=drawColor,line width= 0.4pt,line join=round,line cap=round] (147.37,340.63) circle (  2.25);

\path[draw=drawColor,line width= 0.4pt,line join=round,line cap=round] (156.68,340.54) circle (  2.25);

\path[draw=drawColor,line width= 0.4pt,line join=round,line cap=round] (168.07,326.89) circle (  2.25);

\path[draw=drawColor,line width= 0.4pt,line join=round,line cap=round] (177.38,312.75) circle (  2.25);

\path[draw=drawColor,line width= 0.4pt,line join=round,line cap=round] (187.63,298.10) circle (  2.25);

\path[draw=drawColor,line width= 0.4pt,line join=round,line cap=round] (198.09,292.96) circle (  2.25);

\path[draw=drawColor,line width= 0.4pt,line join=round,line cap=round] (208.33,282.47) circle (  2.25);

\path[draw=drawColor,line width= 0.4pt,line join=round,line cap=round] (219.42,268.28) circle (  2.25);

\path[draw=drawColor,line width= 0.4pt,line join=round,line cap=round] (229.55,253.60) circle (  2.25);

\path[draw=drawColor,line width= 0.4pt,line join=round,line cap=round] (239.70,259.15) circle (  2.25);

\path[draw=drawColor,line width= 0.4pt,line join=round,line cap=round] (249.57,252.08) circle (  2.25);

\path[draw=drawColor,line width= 0.4pt,line join=round,line cap=round] (259.56,247.53) circle (  2.25);

\path[draw=drawColor,line width= 0.4pt,line join=round,line cap=round] (270.27,245.25) circle (  2.25);

\path[draw=drawColor,line width= 0.4pt,line join=round,line cap=round] (280.27,233.03) circle (  2.25);

\path[draw=drawColor,line width= 0.4pt,line join=round,line cap=round] (290.52,222.30) circle (  2.25);

\path[draw=drawColor,line width= 0.4pt,line join=round,line cap=round] (300.85,212.46) circle (  2.25);

\path[draw=drawColor,line width= 0.4pt,line join=round,line cap=round] (311.12,206.62) circle (  2.25);

\path[draw=drawColor,line width= 0.4pt,line join=round,line cap=round] (321.21,200.48) circle (  2.25);

\path[draw=drawColor,line width= 0.4pt,line join=round,line cap=round] (331.48,190.02) circle (  2.25);

\path[draw=drawColor,line width= 0.4pt,line join=round,line cap=round] (341.75,184.01) circle (  2.25);

\path[draw=drawColor,line width= 0.4pt,line join=round,line cap=round] (352.00,171.17) circle (  2.25);

\path[draw=drawColor,line width= 0.4pt,line join=round,line cap=round] (362.30,162.08) circle (  2.25);

\path[draw=drawColor,line width= 0.4pt,line join=round,line cap=round] (372.52,154.06) circle (  2.25);

\path[draw=drawColor,line width= 0.4pt,line join=round,line cap=round] (382.78,144.33) circle (  2.25);

\path[draw=drawColor,line width= 0.4pt,line join=round,line cap=round] (393.01,136.37) circle (  2.25);

\path[draw=drawColor,line width= 0.4pt,line join=round,line cap=round] (403.23,127.18) circle (  2.25);

\path[draw=drawColor,line width= 0.4pt,line join=round,line cap=round] (413.49,119.33) circle (  2.25);

\path[draw=drawColor,line width= 0.4pt,line join=round,line cap=round] (423.73,111.14) circle (  2.25);

\path[draw=drawColor,line width= 0.4pt,line join=round,line cap=round] (433.98,103.30) circle (  2.25);

\path[draw=drawColor,line width= 0.4pt,line join=round,line cap=round] (444.21, 94.45) circle (  2.25);

\path[draw=drawColor,line width= 0.4pt,line join=round,line cap=round] (454.46, 84.11) circle (  2.25);

\path[draw=drawColor,line width= 0.4pt,line join=round,line cap=round] (464.71, 75.85) circle (  2.25);
\definecolor{drawColor}{RGB}{0,0,0}

\path[draw=drawColor,line width= 0.4pt,line join=round,line cap=round] (395.89,456.69) rectangle (480.69,420.69);
\definecolor{drawColor}{RGB}{0,0,255}

\path[draw=drawColor,line width= 0.4pt,line join=round,line cap=round] (398.59,444.69) -- (416.59,444.69);
\definecolor{drawColor}{RGB}{255,0,0}

\path[draw=drawColor,line width= 0.4pt,line join=round,line cap=round] (398.59,432.69) -- (416.59,432.69);
\definecolor{drawColor}{RGB}{0,0,255}

\path[draw=drawColor,line width= 0.4pt,line join=round,line cap=round] (407.59,444.69) circle (  2.25);
\definecolor{drawColor}{RGB}{255,0,0}

\path[draw=drawColor,line width= 0.4pt,line join=round,line cap=round] (407.59,432.69) circle (  2.25);
\definecolor{drawColor}{RGB}{0,0,0}

\node[text=drawColor,anchor=base west,inner sep=0pt, outer sep=0pt, scale=  1.00] at (425.59,441.25) {Sample Var};

\node[text=drawColor,anchor=base west,inner sep=0pt, outer sep=0pt, scale=  1.00] at (425.59,429.25) {Asymptotic};
\end{scope}
\end{tikzpicture}
}
%     \caption{Sample variance of $\hat\theta_q$ (shown in \textcolor{red}{red}) plotted against the expected variance according to the asymptotic approximation (shown in \textcolor{blue}{blue}) on log-scales for both the $x$ and $y$ axes. The approximate variance obtained from the delta method approaches the sample variance as $B$ grows large; although the approximate variance tends to underestimate the variance (as indicated by the sample variance) for small $B$. Note the differences in the $x$ and $y$ axes; $\Var(\hat\theta_q)$ tends to grow disproportionately as $q$ becomes small.}
%     \label{fig:asymptotics-plots}
% \end{figure}

\section{Simulation Experiments}

\begin{table}[ht]
    \centering
    \caption{Point estimator performance, ordered by MSE (error = $0.5*\sigma$)}
    
\begin{tabular}{lrrrr}
\toprule
\multicolumn{1}{l}{Method} & \multicolumn{1}{c}{MSE} & \multicolumn{1}{c}{Bias} & \multicolumn{1}{c}{Variance} & \multicolumn{1}{c}{Average Runtime} \\
 & (000's years) & (years) & (000's years) & (seconds)\\
\midrule
BA-MLE & 244 & -22 & 244 & 0.00003\\
GRIWM-BA (q=0.5) & 245 & 95 & 236 & 13.88254\\
STRAUSS & 246 & -23 & 245 & 0.00002\\
SI-UGM & 251 & 117 & 237 & 2.33059\\
MINMI & 253 & 119 & 239 & 0.00047\\
\addlinespace
MLE & 428 & 455 & 221 & 0.00002\\
SI-RM & 428 & 455 & 221 & 0.06072\\
GRIWM (q=0.05) & 1276 & -993 & 291 & 2.35887\\
\bottomrule
\end{tabular}

    \label{tab:table-sim-exp-point-error0.5}
\end{table}

\begin{table}[ht]
    \centering
    \caption{Point estimator performance, ordered by MSE (error = $2*\sigma$)}
    
\begin{tabular}{lrrrr}
\toprule
\multicolumn{1}{l}{Method} & \multicolumn{1}{c}{MSE} & \multicolumn{1}{c}{Bias} & \multicolumn{1}{c}{Variance} & \multicolumn{1}{c}{Average Runtime} \\
 & (000's years) & (years) & (000's years) & (seconds)\\
\midrule
MINMI & 492 & 27 & 492 & 0.00071\\
SI-UGM & 502 & 65 & 498 & 1.68008\\
GRIWM-BA (q=0.5) & 505 & -278 & 428 & 13.94220\\
MLE & 507 & 254 & 443 & 0.00002\\
SI-RM & 507 & 254 & 443 & 0.05993\\
\addlinespace
BA-MLE & 543 & -234 & 489 & 0.00002\\
Strauss & 554 & -248 & 493 & 0.00002\\
GRIWM (q=0.05) & 2507 & -1408 & 526 & 2.36410\\
\bottomrule
\end{tabular}

    \label{tab:table-sim-exp-point-error2}
\end{table}

Interestingly, as measurement error increased in magnitude, \textbf{MLE became more accurate}, likely due to the fact that measurement error is negatively skewed since they cannot be greater than $K-\theta$ (see \autoref{fig: minmi_integral}). This means that as we increase the size of our measurement errors, the observed minimum becomes more likely to be less than $\theta$, which tends to cancel out with the positive bias of the MLE and produce fairly good estimates. Logically, this means the \textbf{MLE's estimates are much more varied}, which we can clearly see in \autoref{tab:table-sim-exp-point-error2} as it has the greatest sample variance.

\begin{table}[ht]
    \centering
    \caption{Confidence Interval Width}
    
\begin{tabular}{lrrrr}
\toprule
\multicolumn{1}{c}{Method} & \multicolumn{4}{c}{Average Width} \\
 & 0*$\sigma$ & 0.5*$\sigma$ & 1*$\sigma$ & 2*$\sigma$\\
\midrule
SI-RM & 2054.55 & 2375.39 & 2500.86 & 2459.19\\
SI-UGM & 1961.14 & 2091.09 & 2964.30 & 2351.53\\
MINMI & 1917.16 & 2077.86 & 2932.73 & 2329.95\\
GRIWM-corrected & 0.00 & 548.08 & 1949.67 & 1047.99\\
GRIWM & 0.00 & 608.33 & 2163.50 & 1162.65\\
\bottomrule
\end{tabular}

    \label{tab:table-sim-exp-width}
\end{table}

\begin{table}[ht]
    \centering
    \caption{Confidence Interval Run Time}
    
\begin{tabular}{lrrrr}
\toprule
\multicolumn{1}{c}{Method} & \multicolumn{4}{c}{Average Runtime} \\
 & 0*$\sigma$ & 0.5*$\sigma$ & 1*$\sigma$ & 2*$\sigma$\\
\midrule
SI-RM & 0.0564 & 0.0607 & 0.0599 & 0.0599\\
SI-UGM & 4.7066 & 2.3306 & 1.6801 & 1.9374\\
MINMI & 0.0000 & 0.0013 & 0.0021 & 0.0014\\
GRIWM-BA (q=0.5) & 13.8988 & 13.8825 & 13.9422 & 13.9010\\
GRIWM (q=0.05) & 2.3355 & 2.3589 & 2.3641 & 18.1072\\
\bottomrule
\end{tabular}

    \label{tab:table-sim-exp-runtime}
\end{table}

\begin{figure}[ht]
    \centering
    \includesvg[inkscapelatex=false, width=\linewidth]{figures/plot-sim-exp-point-est-grid.svg}
    \caption{}
    \label{fig:sim-exp-grid}
\end{figure}

% \begin{figure}[ht]
%     \centering
%     \includesvg[inkscapelatex=false, width=\linewidth]{figures/plot-sim-exp-point-est-Bias.svg}
%     \caption{}
%     \label{fig:sim-exp-bias}
% \end{figure}

% \begin{figure}[ht]
%     \centering
%     \includesvg[inkscapelatex=false, width=\linewidth]{figures/plot-sim-exp-point-est-MSE.svg}
%     \caption{}
%     \label{fig:sim-exp-mse}
% \end{figure}

% \begin{figure}[ht]
%     \centering
%     \includesvg[inkscapelatex=false, width=\linewidth]{figures/plot-sim-exp-point-est-Runtime.svg}
%     \caption{}
%     \label{fig:sim-exp-runtime}
% \end{figure}

% \begin{figure}[ht]
%     \centering
%     \includesvg[inkscapelatex=false, width=\linewidth]{figures/plot-sim-exp-point-est-Variance.svg}
%     \caption{}
%     \label{fig:sim-exp-varianc}
% \end{figure}

\section{Application}

\begin{figure}[ht]
    \centering
    \includesvg[inkscapelatex=false, width=\linewidth]{figures/applications-hists.svg}
    \caption{Histograms of the four megafauna species' datasets according to default \texttt{ggplot2} binning. Values of $K$ and $\bm{\theta}$ were chosen by inspection of these histograms. From left to right: the cave bear, cave hyena, Eurasian woolly mammoth, and steppe bison.}
    \label{fig:applications-histograms}
\end{figure}

