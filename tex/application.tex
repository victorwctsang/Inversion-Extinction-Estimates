
In this chapter, we discuss the results of applying the proposed MINMI and SI-RM estimators to real datasets to demonstrate their use to palaeobiologists, comparing their estimates to estimates found using existing methods such as: the uncorrected version of GRIWM with $q=0.05$, the bias-adjusted version of GRIWM (GRIWM-BA) with $q=0.5$, and the SI-UGM estimator. We investigate the performance of point estimates and confidence intervals constructed by the various methods and compare their performance and ease of use to the results from the simulation studies.

We applied these three methods to 4 fossil records of megafauna believed to have gone extinct in the Late Pleistocene era/early Holocene period: the steppe bison (\textit{Bison Priscus}), the cave bear (\textit{Ursus spelaeus}), the cave hyena (Crocuta spelaea) and the Eurasian woolly mammoth (\textit{Mammuthus primigenius}). We also investigate the results for the cave bear more closely to place our methods in context of the literature.

\section{Setup}

Under the \hyperref[model: measurement-error]{$\varepsilon$-model}, some choices of values need to made. One such choice is the value of $K$ --- since the speciation and invasion dates of the species are not known, $K$ was set to the age of the oldest observation in each dataset, reducing the number of observations used for estimation to $n-1$. Another choice is the amount of measurement error variation, which was set by taking the average standard error in each dataset. These have been summarised in \autoref{tab:table-applications-Ks}, along with the vector of candidate values used in SI-UGM estimates for each dataset.
\begin{table}[ht]
    \centering
    \caption{Choices of $K$, $\sigma$, and $\bm{\theta}$ for each dataset.}
    
\begin{tabular}{lrrl}
\toprule
Dataset & $K$ & $\sigma$ & $\bm{\theta}$\\
\midrule
Cave Bear       & 34{\small,}363.25 & 668.58            & [9{\small,}985.94 to 32{\small,}911.93]\\
Cave Hyena      & 52{\small,}953.75 & 1{\small,}117.92  & [-34{\small,}279.5 to 48{\small,}202.49]\\
Eurasian Mammoth& 25{\small,}846.75 & 282.42            & [-31{\small,}380.52 to 23{\small,}399.6]\\
Steppe Bison    & 55{\small,}315.25 & 1{\small,}270.52  & [-50{\small,}649.96, to 55{\small,}148.06]\\
\bottomrule
\end{tabular}

    \label{tab:table-applications-Ks}
    \vspace{-4mm}
\end{table}

One difficulty when applying the SI-UGM estimator is the choice of $\bm{\theta}$, the vector of candidate values for $\theta$. Broadly speaking, as long as $\bm{\theta}$ has a wide enough range with enough values such that all plausible values of theta can be covered, we can be confident of getting reasonably accurate estimate (albeit at the cost of requiring more computation). However, there is currently no precise process for choosing this range and number of values. For our purposes, intervals were chosen by taking integer years from 2 standard deviations down from the youngest fossil observation up to 1 standard deviation above the youngest fossil. This was selected to account for the sampling error in fossil datasets, since extinction times are almost certainly more recent than the most recent observation in the fossil record (see \autoref{tab:table-applications-Ks}).

The proposed estimators (MINMI and SI-RM) also have some considerations when applying them to datasets. Similar to the simulation studies, we aimed for estimates with Monte Carlo error variance less than 20\% of $\sigma^2$. The number of Monte Carlo samples used for MINMI have been summarised in \autoref{tab:table-applications-minmi-Bs}, while the choice of starting values, step length constants, and stopping criteria in SI-RM followed the choices outlined previously.
\begin{table}[ht]
    \centering
    \caption{Number of Monte Carlo samples ($B$) used in each scenario for MINMI estimates of the lower ($q=0.025$) and upper ($q=0.975$) endpoints of 95\% confidence intervals. Point estimates are found at $q = 0.5$.}
    \vspace{4mm}
    
\begin{tabular}{lrrr}
\toprule
& \multicolumn{3}{c}{$B$} \\
Dataset & $q = 0.025$ & $q = 0.5$ & $q = 0.975$\\
\midrule
Cave Bear & 4 & 3 & 2\\
Cave Hyena & 3 & 2 & 2\\
Eurasian Mammoth & 4 & 3 & 2\\
Steppe Bison & 6 & 4 & 2\\
\bottomrule
\end{tabular}

    \label{tab:table-applications-minmi-Bs}
\end{table}

\section{Discussion}
All three methods produced comparable confidence intervals, with similar widths, ranges, and point estimates. These results further confirm the results from the simulation experiments, with the MINMI, SI-RM, and SI-UGM estimators producing similar point estimates and confidence intervals while both the uncorrected and corrected versions of GRIWM produced narrower intervals and substantially different point estimates. Another observation of note is the width of the intervals, which tend to narrow as the sample sizes increase: the intervals for the Steppe Bison, where $n = 13$, are substantially wider than those for the Eurasian Woolly Mammoth where $n = 194$. This can be interpreted as being due to the availability of more information contributing to more precise estimates, and therefore narrower confidence intervals.
\begin{figure}[ht]
    \centering
    \includesvg[inkscapelatex=false,width=\linewidth]{figures/applications.svg}
    \caption{Extinction time point estimates and confidence intervals found for each of the four megafauna species. From left to right: the steppe bison, cave bear, cave hyena, and Eurasian woolly mammoth.}
    \label{fig:applications-confidence-intervals}
\end{figure}

\section{Case Study: Cave Bear (\textit{Ursus spelaeus})}

To put our results in context, we considered the cave bear as an example of extinct megafauna from the Pleistocene era and compared our results to those obtained by \citet{Baca2016}, who estimated extinction times using a dataset of over 200 fossils. Unfortunately, we were unable to get access to the dataset used in the paper, instead using the dataset from \citet{Cooper2015} containing 30 samples.

Having disappeared towards the end of the last glacial period, the cave bear's causes of extinction and precise extinction time have been in debate over the years. In their 2016 study, \citet{Baca2016} used the eight methods (including the Strauss, McInerny et al, and GRIWM estimators) to construct 95\% confidence intervals for the extinction time, concluding that extinction likely occurred between $25,234 - 24,291$ calendar years before present (BP). The authors also took a more conservative approach, excluding younger dates that had received criticism for not having certain chemical data. This resulted in more conservative estimates of between $27,000 - 26,100$ years BP (\autoref{fig:application-cave-bear}). Both of these estimates coincide with the last glacial period which, along with anatomical evidence pointing to a vegetarian diet, suggests their disappearance was largely due to the scarcity of food during the onset of the last glacial period rather than the result of climate change inflicted habitat loss or over hunting from humans \cite{Pacher2009}.
\begin{figure}[ht]
    \centering
    \includesvg[inkscapelatex=false,width=\linewidth]{figures/plot-application-cave-bear.svg}
    \caption{Confidence intervals found for the cave bear. GRIWM and GRIWM (conservative) intervals are as published in \citet{Baca2016}.}
    \label{fig:application-cave-bear}
\end{figure}

For our analysis, we used a different dataset of cave bear fossils, obtained from \citet{Cooper2015}. This dataset had 30 samples with the youngest sample dated to approximately $27,832$ years BP, compared to the dataset of 207 used by \citet{Baca2016} with the youngest sample dated to approximately $26,000$ years BP. Using the MINMI, SI-RM, and SI-UGM methods, we constructed confidence intervals that suggested extinction occurred between $29,166 - 26,900$ years BP. These results are somewhat comparable to those obtained by \citet{Baca2016}, and are very similar to their conservative estimates. However, the intervals we estimated are wider (\autoref{fig:application-cave-bear}). This can be attributed to the fact that our dataset is significantly smaller --- the smaller sample size results in wider intervals that reflect the increased uncertainty in our estimates, a result supported by our simulation studies. Moreover, our dataset contained older samples compared to the dataset used by \citet{Baca2016}, which likely contributed to our estimates being somewhat older.