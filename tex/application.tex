
In this chapter, we discuss the results of applying the proposed MINMI and SI-RM estimators to real datasets, comparing their estimates to those found using existing methods: the uncorrected version of GRIWM with $q=0.05$, the bias-adjusted version of GRIWM (GRIWM-BA) with $q=0.5$, and the SI-UGM estimator. We investigate the performance of point estimates and confidence intervals constructed by the various methods and compare their performance and ease of use to the results from the simulation studies.

We applied these three methods to 4 fossil records of megafauna believed to have gone extinct in the Late Pleistocene era/early Holocene period: the steppe bison (\textit{Bison Priscus}), the cave bear (\textit{Ursus spelaeus}), the cave hyena (Crocuta spelaea) and the Eurasian woolly mammoth (\textit{Mammuthus primigenius}). We also investigate the results for the cave bear more closely to place our methods in context of the literature.

\section{Setup}

Under the \hyperref[model: measurement-error]{$\varepsilon$-model}, some choices of values need to made. One such choice is the value of $K$ --- since the speciation and invasion dates of the species are not known, $K$ was set to the age of the oldest observation in each dataset, reducing the number of observations used for estimation to $n-1$. Another choice is the amount of measurement error variation, which was set by taking the average standard error in each dataset. These have been summarised in \autoref{tab:table-applications-Ks}.

One difficulty when applying the SI-UGM estimator is the choice of $\bm{\theta}$, the vector of candidate values for $\theta$. Broadly speaking, as long as $\bm{\theta}$ has a wide enough range with enough values such that all plausible values of theta can be covered, we can be confident of getting reasonably accurate estimate (albeit at the cost of requiring more computation). However, there is currently no precise process for choosing this range and number of values. For the purpose of illustrating how to apply this method, we select $\bm{\theta}$ using the same process as the simulation studies, setting it equal to a vector of length 5{\small,}000 with the elements taken from the set of equally spaced numbers from 10 standard deviations below the BA-MLE to 2 standard deviations above. (see \autoref{tab:table-applications-Ks}).
\begin{table}[ht]
    \centering
    \vspace{-6mm}
    \caption{Choices of $K$, $\sigma$, and $\bm{\theta}$ for each dataset.}
    
\begin{tabular}{lrrl}
\toprule
Dataset & $K$ & $\sigma$ & $\bm{\theta}$\\
\midrule
Cave Bear & 34363.25 & 668.58 & {}[9985.94, 32911.93]\\
Cave Hyena & 52953.75 & 1117.92 & {}[-34279.5, 48202.49]\\
Eurasian Mammoth & 25846.75 & 282.42 & {}[-31380.52, 23399.6]\\
Steppe Bison & 55315.25 & 1270.52 & {}[-50649.96, 55148.06]\\
\bottomrule
\end{tabular}

    \label{tab:table-applications-Ks}
    \vspace{-4mm}
\end{table}

When choosing the number of Monte Carlo samples, $B$, to use in the MINMI method, we aimed for a $B > 100$ that would give estimates with Monte Carlo error variance less than 20\% of $\sigma^2$. However, when solving for $B$ using \autoref{eqn:minmi-optimal-b}, all values found were less than 10 for the four datasets. As such, 100 Monte Carlo samples were used in each application.

Finally, 1{\small,}000 iterations were used for the SI-RM estimator. Initially, we attempted to use a similar process to MINMI, setting a target variance $A$ and iterating until the estimated SI-RM variance was less than $A$. However, this resulted in early termination of the RM process, which would last only 2 steps across all four applications for $A  = 0.2*\sigma$, and only 5 steps for $A = 0.1*\sigma$. This suggests that our estimate of the variance at iteration $i$ is inexact at small $i$. Ultimately, due to time constraints we were unable to resolve this issue, and arbitrarily chose to use 1000 iterations, at which point we identified convergence.

\section{Results}

When applying the five methods to the four datasets, we obtained similar results to the simulation studies. The MINMI, SI-RM, and SI-UGM methods produced comparable confidence intervals, with similar widths, ranges, and point estimates. On the other hand, the two GRIWM implementations produced narrower intervals with smaller point estimates than the other three methods.
\begin{figure}[ht]
    \centering
    \includesvg[inkscapelatex=false,width=\linewidth]{figures/applications.svg}
    \caption{Extinction time point estimates and confidence intervals found for each of the four megafauna species. Each dataset's fossils are shown as cross marks.}
    \label{fig:applications-confidence-intervals}
\end{figure}

The estimated extinction times found by MINMI, SI-RM, and SI-UGM tend to overlap with each their respective fossil records, which seems inconsistent with our intuition about sampling error. One example of this is the cave bear (bottom left in \autoref{fig:applications-confidence-intervals}). However, this is most likely due to the measurement error variation being fairly large relative to the observed range: since the measurement errors are less than $K-\theta$, the joint density is skewed towards negative values of measurement error and it is more likely that the fossil ages are younger than their true ages (\autoref{fig: minmi_integral}). Thus, we can expect estimates under the \hyperref[model: measurement-error]{$\varepsilon$-model} to overlap more with the observed fossil record when measurement error variation is larger.

This can be seen when comparing the results for the cave bear vs. the Eurasian woolly mammoth. Looking at \autoref{fig:applications-confidence-intervals}, we see that the confidence intervals for the cave bear overlap with more of the dataset compared to the mammoth results. When considering the values of $\sigma$, we see that it represents approximately 10\% of the range for the cave bear, while only representing 1\% for the Eurasian mammoth (\autoref{tab:applications-sigma-range-ratios}), which explains the difference.
\begin{table}[ht]
    \centering
    \caption{Measurement error variation relative to the length of the observed fossil recovery interval $K - X_{(1)}$.}
    
\begin{tabular}{lrrl}
\toprule
Dataset & $\sigma$ & $K$ & $\sigma/K$\\
\midrule
Eurasian Mammoth & 282.4200 & 14969.0 & 1.89%\\
Cave Hyena & 1117.9241 & 22253.5 & 5.02%\\
Steppe Bison & 1270.5192 & 21191.5 & 6.00%\\
Cave Bear & 668.5806 & 6023.5 & 11.10%\\
\bottomrule
\end{tabular}

    \label{tab:applications-sigma-range-ratios}
    \vspace{-4mm}
\end{table}

Another observation of note is the different widths of the intervals for each dataset, which tend to narrow with decreasing measurement error variation $\sigma$, as well as increasing sample size. For instance, the Eurasian Mammoth ($\sigma = 282.42$; $n = 202$) has narrower intervals than the Cave Bear ($\sigma = 668.58$; $n = 30$). \autoref{fig:application-widths-lineplot} shows these relationships. This trend is within expectations, as a smaller measurement error variation intuitively means more precise estimates, and we can interpret a larger sample size as having ``more information", which also contributes to more precise estimates.
\begin{figure}[ht]
    \centering
    \includesvg[inkscapelatex=false,width=0.8\linewidth]{figures/plot-application-widths-samplesize.svg}
    \caption{Confidence interval widths plotted against measurement error variation (top) and against sample size (bottom). Notice that width is positively correlated with measurement error variation, while it is negatively correlated with sample size. These are in line with our expectations in regards to precision and the availability of information.}
    \label{fig:application-widths-lineplot}
\end{figure}

\section{Case Study: Cave Bear (\textit{Ursus spelaeus})}

To put our results in context, we considered the cave bear as an example of extinct megafauna from the Pleistocene era and compared our results to those obtained by \citet{Baca2016}, who estimated extinction times using a dataset of over 200 fossils. Unfortunately, we were unable to get access to the dataset used in the paper, instead using the dataset from \citet{Cooper2015} containing 30 samples.

Having disappeared towards the end of the last glacial period, the cave bear's causes of extinction and precise extinction time have been in debate over the years. In their 2016 study, \citet{Baca2016} used the original implementation of GRIWM estimator with threshold probability $q = 0.05$ to construct 95\% confidence intervals for the extinction time, concluding that extinction likely occurred between $25{\small,}234 - 24{\small,}291$ calendar years before present (BP). The authors also considered a more conservative approach, excluding younger dates that were missing certain chemical data. This resulted in more conservative estimates of between $27{\small,}007 - 26{\small,}117$ years BP \cite{Baca2016}.

For our analysis, we used a different dataset of cave bear fossils, obtained from \citet{Cooper2015}. This dataset had 30 samples with the youngest sample dated to approximately $27{\small,}832$ years BP, compared to the dataset of 207 used by \citet{Baca2016} with the youngest sample dated to approximately $26{\small,}000$ years BP. Using the MINMI, SI-RM, and SI-UGM methods, we constructed confidence intervals that suggest extinction occurred between $29{\small,}166 - 26{\small,}900$ years BP.

The intervals estimated by \citet{Baca2016} using GRIWM are narrower than those estimated by the MINMI and SI-RM methods (\autoref{tab:case-study-intervals}). One reason is the fact that our dataset is significantly smaller --- a smaller sample size means estimates have greater uncertainty, and thus we can expect wider intervals. Additionally, our simulation study results are consistent with this application, as the estimates using both the original and bias-adjusted implementations of GRIWM consistently yielded narrower intervals.
\begin{table}[ht]
    \centering
    \caption{Table of estimated intervals and widths for the cave bear. Note that \citet{Baca2016} used the original implementation of GRIWM with $q=0.05$ while the ``conservative" estimate was found using a subset of their data. Estimates for MINMI, SI-RM, and GRIWM-BA ($q=0.5$) were found using the dataset from \citet{Cooper2015}.}
    
\begin{tabular}{lll}
\toprule
Method & 95\% CI & Width\\
\midrule
SI-RM & 27,060 - 29,176 & 2,116\\
MINMI & 26,985 - 29,176 & 2,191\\
GRIWM-BA (q=0.5) & 26,241 - 28,129 & 1,888\\
Baca et al (conservative) & 26,117 - 27,007 & 890\\
Baca et al & 24,291 - 26,081 & 1,790\\
\bottomrule
\end{tabular}

    \label{tab:case-study-intervals}
    \vspace{-4mm}
\end{table}

Another difference in our results is the interval estimated, as the periods estimated by \citet{Baca2016} are clearly more recent than those estimated by MINMI and SI-RM (\autoref{fig:application-cave-bear}). This is likely due to \citet{Baca2016} having used the original implementation of GRIWM with $q = 0.05$, which we found to be downwards biased due to the bias present introduced by the recovery rate from \citet{Mcinerny2006} as well as the choice of threshold probability $q=0.05$. Another reason is related to the datasets used: the youngest sample in the dataset used by \citet{Baca2016} was approximately 1{\small,}000 years more recent than the youngest sample in the dataset we used, contributing to their estimated confidence intervals being more recent.
\begin{figure}[ht]
    \centering
    \includesvg[inkscapelatex=false,width=\linewidth]{figures/plot-application-cave-bear.svg}
    \caption{Confidence intervals found for the cave bear, where the \citet{Cooper2015} dataset with $n=30$ was used for the MINMI, SI-RM, and GRIWM-BA methods. The 30 samples from \citet{Cooper2015} are shown as crosses. The intervals for the \citet{Baca2016} results were taken from the original paper.}
    \label{fig:application-cave-bear}
\end{figure}