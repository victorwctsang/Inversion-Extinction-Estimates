
In this chapter, we discuss the results of applying the proposed MINMI and SI-RM estimators to real datasets to demonstrate their use to paleobiologists, comparing these results to those of the SI-UGM estimator. Our goal is to see if these methods produce estimates and confidence intervals within expectations, given their performance on synthetic datasets with known conditions.

We applied these three methods to 4 fossil records of megafauna believed to have gone extinct in the Late Pleistocene era/early Holocene period: the steppe bison (\textit{Bison Priscus}), the cave bear (\textit{Ursus spelaeus}), the cave hyena (Crocuta spelaea) and the Eurasian woolly mammoth (\textit{Mammuthus primigenius}). We also investigate the results for the cave bear more closely to place our methods in context of the literature.

Since the speciation and invasion dates of the species are not known, we instead chose a value of $K$ for each dataset representing some upper bound on the observations. In the absence of better information, $K$ was set to the start of the last bin in histograms of the data. Extinction time estimates were then found using the data after excluding observations older than this value of $K$. For instance, the value of $K$ for the cave bear was chosen to be $34,000$ by inspecting the histogram in \autoref{fig:applications-histograms}.

A similar process was used to set the input vector of hypothesised values $\bm{\theta}$ for the SI-UGM estimator, which is assumed to cover the entire interval of possible values for the $95\%$ confidence interval. For example, $\bm{\theta}$ for the cave bear was chosen to be all years between $22,000 - 29,000$ years BP.
\begin{figure}[ht]
    \centering
    \includesvg[inkscapelatex=false, width=\linewidth]{figures/applications-hists.svg}
    \caption{Histograms of the four megafauna species' datasets according to default \texttt{ggplot2} binning. Values of $K$ and $\bm{\theta}$ were chosen by inspection of these histograms. From left to right: the cave bear, cave hyena, Eurasian woolly mammoth, and steppe bison.}
    \label{fig:applications-histograms}
\end{figure}

All three methods produced comparable confidence intervals, with similar widths, ranges, and point estimates. These results further confirm the results from the simulation experiments, as the SI-UGM estimator tended to produce wider confidence intervals for the extinction time. Another observation of note is the width of the intervals, which tend to narrow as the sample sizes increase: the intervals for the Steppe Bison, where $n = 13$, are substantially wider than those for the Eurasian Woolly Mammoth where $n = 194$. This can be interpreted as being due to the availability of more information contributing to more precise estimates, and therefore narrower confidence intervals. 
\begin{figure}[ht]
    \centering
    \includesvg[inkscapelatex=false,width=\linewidth]{figures/applications.svg}
    \caption{Extinction time point estimates and confidence intervals found for each of the four megafauna species. From left to right: the steppe bison, cave bear, cave hyena, and Eurasian woolly mammoth.}
    \label{fig:applications-confidence-intervals}
\end{figure}

\section{Case Study: Cave Bear (\textit{Ursus spelaeus})}

To place our results in context, we consider the cave bear as an example of extinct megafauna from the Pleistocene era. Having disappeared towards the end of the last glacial period, the causes of extinction and precise extinction time has been in debate over the years. In a 2016 study by \citet{Baca2016}, the authors analysed a dataset of over 200 fossil samples of cave bears to construct 95\% confidence intervals for the extinction time, estimating it to have occurred between $26,100 - 24,300$ cal. years before present (BP). The authors also took a more conservative approach, excluding younger dates to obtain an earlier confidence interval of between $27,000 - 26,100$ years BP. These extinction times coincide with the last glacial period which, along with anatomical evidence pointing to a vegetarian diet, suggests their disappearance was largely due to the scarcity of food during the onset of the last glacial period rather than the result of climate change inflicted habitat loss or overhunting from humans \cite{Pacher2009}. 

For this thesis, we estimated extinction times from a dataset of 30 cave bear fossils using the MINMI, SIRM, and SI-UGM methods, constructing confidence intervals that suggested extinction occurring roughly between $29,000 - 27,000$ cal. years BP. These results are somewhat comparable to those obtained by \citet{Baca2016}, as the extinction times belong to the same part of the last glacial period and hence support the same conclusion that changes in the climate from the onset of an ice age reduced the availability of plant life for cave bears to feed upon, triggering their extinction. However, it is clear that the intervals we estimated are substantially wider and more biased in comparison. This can be attributed to the fact that our dataset is both significantly smaller and  more biased in comparison: the smaller sample size results in wider intervals that reflect the increased uncertainty in our estimates, and as \autoref{fig:applications-histograms} shows, the youngest sample in our dataset is dated to approximately $28,000$ year BP, explaining the positive bias in our results. Hence, although the results for the cave bear obtained by our estimation methods may not necessarily reflect the true extinction times due to the quality of our dataset, we believe these results nonetheless demonstrate the performance of the MINMI and SIRM estimators, showing that they contend with existing methods and produce reliable and robust estimates that are consistent with the literature.