
Let the random variables $\bm{X} = [X_1, \dots, X_n]^\top$ denote a species' fossil record, where each $X_i$ is measured in the number of years before present (BP) and $X_1$ is the most recent (smallest) fossil time such that $X_i < X_{i+1}$. Let $\theta$ denote the unknown extinction time and let $K$ be a known constant representing the earliest possible time of observation. This may be interpreted as the speciation or invasion date; however, in practice, these dates are often unknown. As such, $K$ can be any arbitrarily selected upper bound as long as it is between the speciation date and extinction date.

Our goal is to estimate the population parameter $\theta$. In this chapter, we will review assumptions that are commonly used in the literature and formulate two models under which various methods can be constructed. These models lay the foundation for review of existing methods in \autoref{chap: existing-methods} and methods proposed in \autoref{chap: proposed-methods}.

\section{$\delta$ Model}\label{section: delta-model}

A common approach is to assume that fossils are uniform and independently distributed over a fixed interval, where measurement error from radiocarbon dating and calibration is negligible relative to sampling error.

We formulate this as the $\delta$-Model, where fossils are a realisation of a homogeneous Poisson counting process over interval $[\theta, K]$ and measurement errors are thought of as following the Dirac delta distribution, where the density function $f(e)$ is 0 for all $e$ except at $e=0$.
\begin{model}[$\delta$ Model]\label{model: no-measurement-error}
    Let the number of fossils be generated by a Poisson counting process with constant fossil recovery rate $\lambda$. Then:
    \begin{align*}
        n   &\sim \textrm{Pois}(\lambda(K-\theta)) \\
        X_i &\overset{i.i.d}{\sim} \mathcal{U}[\theta, K] \\
        G_i &\overset{i.i.d}{\sim} \textrm{Exp}(\lambda)
    \end{align*}
    for $i = 2, 3, \dots, n$, where $\theta$ is the unknown parameter of interest, $K$ is a known constant, and $G_i = X_{i+1} - X_i$ is the time gap between fossils.
\end{model}

The uniformity assumption in this model implies that fossils are also perfectly preserved and recovered, which is not true in general. This is because the abundance of a species can be expected to dwindle towards the beginning (speciation) and end (extinction) times \cite{Lee2010, WangMarshall2016}.

Although palaeontologists agree that uniformity is a typically invalid assumption, the majority of analyses to date still use these ``first-generation" methods (a term coined by \citet{WangMarshall2016} identifying methods that assume uniformity). \textbf{This is because of a lack of better alternatives}: although there are methods that infer recovery rates from data, the required quantitative knowledge of recovery rates are often unavailable, making them inapplicable. Furthermore, the benefits of clarity, simplicity, and convenience continue to make these ``first-generation" methods more appealing than the available alternatives.

Moreover, assuming negligible variation from radiocarbon dating and the calibration process is not always valid. Similar to the uniformity assumption, assuming that measurement error variation is negligible is the most direct and most common approach, with various methods in the literature making use of this assumption \cite{Mcinerny2006, Solow1993, Strauss1989, Wang2016, Weiss1999}. However, there is substantial evidence to show this assumption is not generally applicable, as some fossil records will have dating errors that are comparable to the gaps between fossils \cite{Solow2006}.

\clearpage

\section{$\varepsilon$ Model}\label{section: varepsilon-model}

We now formulate a model under which measurement error is substantial and cannot be neglected, modifying the \hyperref[model: no-measurement-error]{$\delta$-model} to introduce measurement error. Let us assume that the fossil ages $\bm{X}$ are now \textit{unobserved} and that we instead observe $\bm{W} = \bm{X} + \bm{\varepsilon}$, where $\bm{\varepsilon}$ represent measurement errors such that each $\varepsilon_i$ is independently distributed with known density $f(e)$ with mean 0 and constant variance $\sigma^2$.

Instead of assuming $\bm{X} < K$, we now assume $\bm{W} < K$ since $K$ is a known upper bound on the observed fossils. Since $X_i = W_i - \varepsilon_i < K - \varepsilon_i$, the unobserved fossil ages are therefore independent and conditionally uniform on $\bm{\varepsilon}$:
\begin{model}[$\varepsilon$ Model]\label{model: measurement-error}
    Let the number of fossils be generated by a Poisson counting process with constant fossil recovery rate $\lambda$, and let $\varepsilon_i$ be independently and identically distributed with density $f(e)$, mean 0, and constant variance $\sigma^2$. Then:
    \begin{align*}
        n &\sim \textrm{Pois}(\lambda(K-\theta)) \\
        \varepsilon_i &\overset{i.i.d}{\sim} f(e) \quad \E[\varepsilon] = 0, \Var(\varepsilon) = \sigma^2 \\
        X_i | \varepsilon_i &\overset{i.i.d}{\sim} \mathcal{U}[\theta, K-\varepsilon_i] \\
        W_i &= X_i + \varepsilon_i < K
    \end{align*}
    for $i = 2, 3, \dots, n$, where $\theta$ is unknown and $K$ is a known constant.
\end{model}
There are some additional nuances to this model. Since $W = X + \varepsilon < K$ and $X > \theta$, this implies that $\varepsilon < K - \theta$. Hence, in constructing a joint density for $(X, \varepsilon)$, we must also condition on $\varepsilon < K - \theta$. We will show how estimates of extinction times and confidence intervals can be constructed by exploiting these assumptions in our proposal of a novel method in \autoref{chap: proposed-methods}.

There still remains a question of how measurement errors are distributed. In this thesis, we will assume that $f(e)$ is normal and centered on zero with a constant $\sigma$ estimated from repeated radiometric  measurements and calibration. This is a fairly common approach in the literature: however, this model can be generalised to any a priori arbitrary measurement error distributions.

Assuming normally distributed radiometric errors is a common approach, as there is some evidence to show that radiocarbon dating errors are approximately normal \cite{Walker2005Quaternary}. However, although radiocarbon dating errors may be normally distributed (see panel 2 of \autoref{fig:variation_sources}), there is substantial evidence to suggest that the errors introduced by \textbf{calibration} curves are non normal (see panel 3 of \autoref{fig:variation_sources}). Moreover, since the same calibration curves are often used to calibrate the dates of fossils in the same dataset, these calibration errors may also be correlated \cite{Ramsey2009, Ramsey2010, Ramsey2013}.
\begin{figure}[ht]
    \centering
    \includegraphics[width=0.7\linewidth]{figures/variation-sources-king.png}
    \caption{Sources of variation when estimating extinction times: 1. The sampling variation (note that the red line indicates the true extinction date and is past the last available fossil); 2. The radiocarbon dating measurement error, approximately normal in the figure; 3. The error introduced by the calibration process mapping radiocarbon dates to calendar dates. Reprinted from \citet{King2020}.}
    \label{fig:variation_sources}
\end{figure}
