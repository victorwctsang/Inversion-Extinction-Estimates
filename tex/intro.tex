
Dot points...
\begin{itemize}
    \item Extinction time estimation is important in paleontology/paleobiology: it helps us understand the causes and timings of extinctions more precisely.
    \item Due to sampling error (e.g. the Signor-Lipps effect), radiometric error, and calibration error, it is difficult to obtain reliable and high-precision estimates of extinction times.
    \item In our thesis, we investigate various existing methods used for estimating extinction times. Depending on how strict their assumptions are, these methods vary in complexity. For example, the Strauss estimator is a relatively simple method that assumes uniformly distributed fossils and no measurement error.
    \item On the other hand, GRIWM (a relatively more recent method) is a good deal more complex because it assumes that there is measurement error. However, it still assumes uniformity.
    \item We propose two new methods based on test-statistic inversion: the Minimum-Statistic Inversion (MINMI) and the Simulated Inversion - Robbins Monro Process (SI-RM) estimators.
    \item \textit{brief explanation of MINMI}
    \item \textit{brief explanation of SI-RM}
    \item Outline of thesis:
    \begin{enumerate}
        \item Chapter 2: We start by reviewing the assumptions and formulating the models under which most existing methods are based on
        \item Chapter 3: Then we review the various methods that currently exist to solve this problem
        \item Chapter 4: We introduce our proposed methods and their statistical properties, as well as considerations when implementing them.
        \item Chapter 5: We run simulation experiments and compare 8 different methods' performances under known conditions.
        \item Chapter 6: We apply our methods to 4 datasets to compare performances. We also do a case study to see if we arrive at similar conclusions as the literature.
        \item Chapter 7: We discuss areas of improvement and future work.
        \item Chapter 8: Concluding chapter.
    \end{enumerate}
\end{itemize}

The exact timing of a species' extinction is crucial to paleobiological understandings, such as species' lineages, migration patterns, and mass-extinctions. However, researchers have long understood that the last observed fossil of a species almost certainly does not represent the true extinction time, an effect known as the ``Signor-Lipps Effect" \cite{Signor1982}. Combined with the presence of both sampling variation and measurement variation, this means that high-precision inferences of extinction times are virtually impossible \cite{Bradshaw2012}. As a result, the causes and mechanisms of various mass extinction events in history continue to be debated today \cite{Saltre2015}.

In this thesis, we focus on methods to estimate the extinction times of megafauna (species with adults weighing over 45 kilograms) that occurred during the late Quaternary extinction between 110,000 and 11,650 years before present. We discuss and assess the validity of assumptions and methods available to paleobiologists for inferring extinction times in the fossil record, and then propose two alternative approaches for solving this problem, both based on test-statistic inversion.

\textcolor{red}{More to come}