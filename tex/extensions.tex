
\begin{enumerate}
    \item \textbf{Relax uniformity assumption}. For example, instead of having $\bm{X}$ as conditionally uniform, we could try sigmoidal or exponential \cite{Bradshaw2012}, or the ``reflected beta distribution" \cite{Wang2016}. This is fairly easy for SIRM, but some maths would be necessary for MINMI.
    \item \textbf{Prove the MINMI stochastically increasing property generally}. We were able to show this is true down to a reasonable constraint (symmetric/unimodal measurement error + small measurement error size relative to the geological time scale), but maybe it could be found in general.
    \item \textbf{Correlated Measurement Error}. Since there is evidence to show that calibrated radiocarbon dates have correlated errors (many fossils are often calibrated using the same calibration curve e.g. IntCal13 or IntCal20) and currently \textbf{no methods in the literature address this}, this would be useful. SI-RM and SI-UGM would be good candidates for this since the simulation model just needs to be adjusted.
    \item \textbf{Test Statistic Selection}. We used the sample minimum as this is the MLE given a uniformity assumption, and it is monotonic with respect to $\theta$. A better statistic could potentially be found, and work could be done to investigate how a ``best" statistic should be found.
    \item \textbf{Estimating the optimal step-length constant for SI-RM}. Garthwaite and Buckland suggested an adaptive step length constant $c$, proportional to the distance from the point estimate of $\theta$. They suggested setting the proportionality constant to be double that of the optimal value for a normally distributed statistic, which they showed gives asymptotically exact results where the distribution of the point estimate is fairly similar to a normal distribution. This could be further refined: for instance, \citet{LlyodBotev2015} suggested a step-length constant without such a normality assumption that could be used to refine the step length constant following a burn-in period.
    \item \textbf{Compare methods on the same datasets}. Due to time constraints, we weren't able to compare MINMI and SI-RM to existing methods on the same dataset. This would give more conclusive comparisons of our methods' performances.
\end{enumerate}